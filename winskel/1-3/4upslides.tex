% 4up version
\documentclass[handout, 12pt, aspectratio=169]{beamer}

%\documentclass[12pt,aspectratio=169]{beamer}

\usefonttheme{professionalfonts}

\mode<handout>
{
\usepackage{pgfpages}
\pgfpagesuselayout{4 on 1}[a4paper,landscape]
}

\usepackage[english]{babel}
\usepackage{luatexja}
\usepackage{luatexja-fontspec}


% for display code
%\usepackage{minted}
%\newfontfamily\hasklig{hasklig}[NFSSFamily=haskligFamily]
%\setminted[haskell]{fontfamily=haskligFamily,
%mathescape,
%       numbersep=5pt
%}

%\usepackage{bcprules}

%\usepackage[T1]{fontenc}

%for lualatex
\usepackage{fontspec}
\setsansfont{CMU Sans Serif}%{Arial}
\setmainfont{CMU Serif}%{Times New Roman}
\setmonofont{CMU Typewriter Text}%{Consolas}

\usepackage{lmodern,amsmath,amssymb,proof}
\usepackage{tikz}
\usetikzlibrary{arrows,positioning}




% color scheme
\definecolor{green}{rgb}{0.0,0.5,0.0}
\definecolor{blue}{rgb}{0.0,0.0,0.7}
\definecolor{red}{rgb}{0.8,0.0,0.0}
\definecolor{lightred}{rgb}{1.0,0.97,0.97}
\definecolor{lightblue}{rgb}{0.95,0.95,1.0}
\definecolor{darkblue}{rgb}{0.3,0.3,0.5}
\definecolor{darkred}{rgb}{0.6,0.0,0.0}
\definecolor{darkorange}{rgb}{0.6,0.2,0.1}
\definecolor{darkgreen}{rgb}{0,0.3,0}

%\definecolor{orenge}{rgb}{0.84, 0.4, 0}
\definecolor{lightgray}{rgb}{0.82, 0.82, 0.82}
\definecolor{lightorange}{rgb}{0.99, 0.83, 0.5}
\definecolor{orange}{rgb}{0.99, 0.33, 0.2}

\newcommand{\IMAGE}[2]{\pgfdeclareimage[#1]{#2}{#2}\pgfuseimage{#2}}

\setbeamerfont{title}{series=\bfseries,size=\Large}
\setbeamercolor{title}{fg=red}
\setbeamerfont{subtitle}{series=\bfseries,size=\LARGE}
\setbeamercolor{subtitle}{fg=red}
\setbeamerfont{author}{series=\bfseries,size=\large}
\setbeamercolor{author}{fg=blue}
\setbeamerfont{institute}{series=\bfseries,size=\large}
\setbeamercolor{institute}{fg=green}

\setbeamertemplate{blocks}[rounded]
\setbeamerfont{block title}{series=\bfseries,family=\sffamily,size=\small\strut}
\setbeamercolor{block title}{bg=darkblue,fg=white}
\setbeamercolor{block body}{bg=blue!4!white}
\setbeamercolor{block title example}{bg=white,fg=darkgreen}
\setbeamercolor{block body example}{bg=white}
\setbeamercolor{block title alerted}{bg=darkred,fg=white}
\setbeamercolor{block body alerted}{bg=red!4!white}
\setbeamercolor{structure}{fg=darkred}
% no fading effect
\makeatletter
\pgfdeclareverticalshading[lower.bg,upper.bg]{bmb@transition}{200cm}{%
  color(0pt)=(lower.bg); color(4pt)=(lower.bg); color(4pt)=(upper.bg)}
\makeatother

% frametitle
\useframetitletemplate{
\begin{centering}
\centerline{\large\bfseries\color{darkblue}\insertframetitle}
\end{centering}
}

% footer  (title  page/pages)
\setbeamertemplate{navigation symbols}{}
\useheadtemplate{\vbox{\vskip8pt}}
\usefoottemplate{\vbox{\vskip2pt\inserttitle\hfil\insertframenumber/\inserttotalframenumber\vskip5pt}}

% enumerate/itemize environment
\newcommand*\tikzboxed[1]{\tikz[baseline=(c.base)]{%
\node[thick,shape=rectangle,draw,inner sep=2pt] (c) {#1};}}
\setbeamertemplate{items}[square]
\setbeamertemplate{enumerate item}{\tikzboxed{\footnotesize\insertenumlabel}}
\setlength{\itemsep}{1ex}

\newcommand{\m}[1]{\mathsf{#1}}
\newcommand{\mi}[1]{\mathit{#1}}
\newcommand{\md}[1]{\mathtt{\textcolor{blue}{#1}}}
\newcommand{\seq}[2][n]{{#2_1},\dots,{#2_{#1}}}
%
\newcommand{\FF}{\mathcal{F}}
\newcommand{\RR}{\mathcal{R}}
\newcommand{\VV}{\mathcal{V}}
\newcommand{\TT}{\mathcal{T}}
\newcommand{\Var}{\mathcal{V}\m{ar}}
%
\newcommand{\app}{\circ}

\newcommand{\Pow}{\mathcal{P}ow}

\newcommand{\empha}[1]{\colorbox{lightorange}{#1}}
\newcommand{\repbc}[2]{\alt<#1>{\empha{#2}}{#2}}
\newcommand{\bcgray}[2]{\alt<#1>{\colorbox{lightgray}{#2}}{#2}}

\title{The Formal Semantics of Programming Languages / Chapter1-3}
\author{Wataru Yachi}
\institute{JAIST}
\date{May 25, 2023}

\begin{document}

\maketitle

\begin{frame}
    \frametitle{Diagonal Argument}
    \begin{block}{Claim}<+->
        A set $X$ and $\Pow(X)$ are never in 1-1 correspondence for any X.
    \end{block}

    \begin{example}<+->
        \begin{itemize}
            \item $X = \{ \repbc{3-}{$1$}, \alt<4->{\empha{$2$}}{2}, \alt<5->{\empha{$3$}}{3} \}$\\
            \item $\Pow(X) = \{ \repbc{3-}{$\phi$}, \repbc{4-}{$\{1\}$}, \repbc{5-}{$\{2\}$}, \bcgray{6-}{$\{3\}$}, \bcgray{6-}{$\{1,2\}$}, \bcgray{6-}{$\{1,3\}$}, \bcgray{6-}{$\{2,3\}$}, \bcgray{6-}{$\{1,2,3\}$} \}$ \\
        \end{itemize}

        \onslide<7->{

        but, if $X$ is infinite\\
        \begin{itemize}
            \item $X = \{1,2,3,4,\cdots\}$\\
            \item $\Pow(X) = \{\phi, \{1\}, \{2\}, \{3\}, \cdots  \}$\\
        \end{itemize}
        dose claim still hold?\\
        }
        \onslide<8->{
        answer is \alert{YES}
        }
    \end{example}
\end{frame}




\begin{frame}
    \frametitle{Diagonal Argument}
    \begin{block}{Claim}<+->
        A set $X$ and $\Pow(X)$ are never in 1-1 correspondence for any X.
    \end{block}

    \begin{proof}<+->
        \onslide<+->{
            Proof by contradiction. Consider a set $X$ and its powerset $\Pow(x)$. \\
        }
        \onslide<+->{
        Let $\theta : X \rightarrow \Pow(X)$
            be a 1-1 correspondence between $X$ and $\Pow(X)$.
        }
        \onslide<+->{Suppose $Y = \{ x \in X \mid x \notin \theta(x) \}$.
            $Y$ is a subset of $X$ and therefore in correspondence with a $y \in X$.
        }
        \onslide<+->{
            So $\theta(y)=Y$. Thus either $y \in Y$ or $y \notin Y$.\\
        }
        \begin{itemize}[<+->]
        \item If $y \in Y$ then $y \notin Y$, because $y \notin \theta(y)$.
        
        \item If $y \notin Y$ then $y \in \theta(y)$, so $y \in Y$.
        \end{itemize}
        \onslide<+->{
            In either case, we have contradiction.
        }
    \end{proof}
\end{frame}



\begin{frame}
    \frametitle{Why it's Called ``Diagonal" Arugument}
        consider following table where $i$th row and $j$th column is placed $1$ if $x_i \in \theta(x_j)$ and $0$ otherwise 

    \begin{table}
        \centering
    \begin{tabular}{c | c c c c c c}
        & $\theta(x_0)$ & $\theta(x_1)$ & $\theta(x_2)$ & $\cdots$ & $\theta(x_j)$ & $\cdots$ \\ \hline
        $x_0$    & \repbc{2-}{$0$} & $1$ & $1$ & $\cdots$ & $1$ & $\cdots$ \\
        $x_1$    & $1$ & \repbc{2-}{$1$} & $0$ & $\cdots$ & $0$ & $\cdots$ \\
        $x_2$    & $0$ & $0$ & \repbc{2-}{$1$} & $\cdots$ & $1$ & $\cdots$ \\
        $\vdots$ & $\vdots$ & $\vdots$ & $\vdots$ & & $\vdots$ & \\
        $x_i$    & $0$ & $1$ & $0$ & $\cdots$ & \repbc{2-}{$1$} & $\cdots$ \\
        $\vdots$ & $\vdots$ & $\vdots$ & $\vdots$ & & $\vdots$ & \\
    \end{tabular}
    \end{table}

    \pause
    in above table, we can define $Y$ by pick up $x_n$ which corresponding cell along diagonal is 0\\
    \pause
    \onslide{\centering
    $Y = \{x_0\}$
    }
\end{frame}

\begin{frame}
    \frametitle{Direct and Inverse Image of Relation}
    \begin{definition}[direct and inverse image]
        Let be  $R:X \times Y$ is a relation, A is a subset of X and B is a subset of Y,
        a set $RA$ and $R^{-1}B$ are defined as follows;
        \begin{itemize}
            \item $RA = \{y \in Y \mid \exists x \in A ((x,y) \in R)\}$,
            \item $R^{-1}B = \{x \in X \mid \exists y \in B ((x,y) \in R)\}$. 
        \end{itemize}
        The set $RA$ is called \alert{direct image} of $A$ under $R$,
        and the set $R^{-1}B$ is called \alert{inverse image} of $B$ under $R$.
            \end{definition}

    \begin{example}<+->
        \begin{itemize}
            \item $X = \{0,1,2,3,4\}$, $Y = \{5,6,7,8,9\}$, $R = \{(0,9), (1,5), (2,6), (3,6)\}$, $A = \{1,3,4\}$, $B = \{5,7,9\}$
            \item $RA = \{5,6\}$, $R^{-1}B = \{0,1\}$
        \end{itemize}
    \end{example}
\end{frame}


\begin{frame}
    \frametitle{Equivalence Relation}
    \begin{definition}[equivalence relation and equivalence class]
        An equivalence relation is a relation $R \subseteq X \times X$ on a set $X$ which is
        \begin{itemize}
            \item reflexive: $\forall x \in X(xRx)$,
            \item symmetric: $\forall x,y \in X(xRy \to yRx)$ and
            \item transitive: $\forall x,y,z \in X((xRy \land yRz) \to xRz)$ .
        \end{itemize}
    \end{definition}
    \begin{example}<+->
        \begin{itemize}
            \item $R = \{(x,y) \in B \times B \mid x \in A \land y \in A\} $ where $A \subseteq B$
            \item congruence of figures
            \item congruence of integers
        \end{itemize}
    \end{example}
\end{frame}

\begin{frame}
    \frametitle{Equivalence Class}
    \begin{definition}[equivalence class]
        If $R$ is an equivalence relation on a set X then ($R-$)equivalence class $\{x\}_R$ of an element $x\in X$
        is defined as follows;
        \[
            \{x\}_R = \{y \in X \mid yRx\}.
        \]
    \end{definition}
    \begin{example}<+->
        \begin{itemize}
            \item $R = \{(x,y) \in \mathbb{Z} \times \mathbb{Z} \mid x \equiv y\pmod 3\}$
            \item $\{1\}_R = \{1, 4, 7, 10, \cdots\}$
        \end{itemize}
    \end{example}
\end{frame}
\end{document}
