% 4up version
%\documentclass[handout, 12pt, aspectratio=169]{beamer}

\documentclass[12pt,aspectratio=169]{beamer}

\usefonttheme{professionalfonts}

\mode<handout>
{
\usepackage{pgfpages}
\pgfpagesuselayout{4 on 1}[a4paper,landscape]
}

\usepackage[english]{babel}
\usepackage{luatexja}
\usepackage{luatexja-fontspec}


% for display code
%\usepackage{minted}
%\newfontfamily\hasklig{hasklig}[NFSSFamily=haskligFamily]
%\setminted[haskell]{fontfamily=haskligFamily,
%mathescape,
%       numbersep=5pt
%}

% for proof tree
%\usepackage{bcprules}

%\usepackage[T1]{fontenc}

%for lualatex
\usepackage{fontspec}
\setsansfont{CMU Sans Serif}%{Arial}
\setmainfont{CMU Serif}%{Times New Roman}
\setmonofont{CMU Typewriter Text}%{Consolas}

\usepackage{mathtools}
\usepackage{lmodern,amsmath,amssymb,proof}
\usepackage{tikz}
\usetikzlibrary{arrows,positioning}



% color scheme
\definecolor{green}{rgb}{0.0,0.5,0.0}
\definecolor{blue}{rgb}{0.0,0.0,0.7}
\definecolor{red}{rgb}{0.8,0.0,0.0}
\definecolor{lightred}{rgb}{1.0,0.97,0.97}
\definecolor{lightblue}{rgb}{0.95,0.95,1.0}
\definecolor{darkblue}{rgb}{0.3,0.3,0.5}
\definecolor{darkred}{rgb}{0.6,0.0,0.0}
\definecolor{darkorange}{rgb}{0.6,0.2,0.1}
\definecolor{darkgreen}{rgb}{0,0.3,0}
%
\newcommand{\IMAGE}[2]{\pgfdeclareimage[#1]{#2}{#2}\pgfuseimage{#2}}

\setbeamerfont{title}{series=\bfseries,size=\Large}
\setbeamercolor{title}{fg=red}
\setbeamerfont{subtitle}{series=\bfseries,size=\LARGE}
\setbeamercolor{subtitle}{fg=red}
\setbeamerfont{author}{series=\bfseries,size=\large}
\setbeamercolor{author}{fg=blue}
\setbeamerfont{institute}{series=\bfseries,size=\large}
\setbeamercolor{institute}{fg=green}

\setbeamertemplate{blocks}[rounded]
\setbeamerfont{block title}{series=\bfseries,family=\sffamily,size=\small\strut}
\setbeamercolor{block title}{bg=darkblue,fg=white}
\setbeamercolor{block body}{bg=blue!4!white}
\setbeamercolor{block title example}{bg=white,fg=darkgreen}
\setbeamercolor{block body example}{bg=white}
\setbeamercolor{block title alerted}{bg=darkred,fg=white}
\setbeamercolor{block body alerted}{bg=red!4!white}
\setbeamercolor{structure}{fg=darkred}
% no fading effect
\makeatletter
\pgfdeclareverticalshading[lower.bg,upper.bg]{bmb@transition}{200cm}{%
  color(0pt)=(lower.bg); color(4pt)=(lower.bg); color(4pt)=(upper.bg)}
\makeatother

% frametitle
\useframetitletemplate{
\begin{centering}
\centerline{\large\bfseries\color{darkblue}\insertframetitle}
\end{centering}
}

% footer  (title  page/pages)
\setbeamertemplate{navigation symbols}{}
\useheadtemplate{\vbox{\vskip8pt}}
\usefoottemplate{\vbox{\vskip2pt\inserttitle\hfil\insertframenumber/\inserttotalframenumber\vskip5pt}}

% enumerate/itemize environment
\newcommand*\tikzboxed[1]{\tikz[baseline=(c.base)]{%
\node[thick,shape=rectangle,draw,inner sep=2pt] (c) {#1};}}
\setbeamertemplate{items}[square]
\setbeamertemplate{enumerate item}{\tikzboxed{\footnotesize\insertenumlabel}}
\setlength{\itemsep}{1ex}

\newcommand{\m}[1]{\mathsf{#1}}
\newcommand{\mi}[1]{\mathit{#1}}
\newcommand{\md}[1]{\mathtt{\textcolor{blue}{#1}}}
\newcommand{\seq}[2][n]{{#2_1},\dots,{#2_{#1}}}
%
\newcommand{\FF}{\mathcal{F}}
\newcommand{\RR}{\mathcal{R}}
\newcommand{\VV}{\mathcal{V}}
\newcommand{\TT}{\mathcal{T}}
\newcommand{\Var}{\mathcal{V}\m{ar}}
%
\newcommand{\app}{\circ}
\def\coloneqq{\mathrel{\mathop:}=}%
\newcommand{\cpair}[1]{\langle #1 \rangle}
\newcommand{\ifthen}[3]{\mathbf{if}\; #1 \; \mathbf{then}\; #2 \; \mathbf{else}\; #3}
\newcommand{\while}[2]{\mathbf{while}\; #1 \; \mathbf{do}\; #2}
\AtBeginDocument{
  \abovedisplayskip     =0.2\abovedisplayskip
  \abovedisplayshortskip=0.2\abovedisplayshortskip
  \belowdisplayskip     =0.2\belowdisplayskip
  \belowdisplayshortskip=0.2\belowdisplayshortskip}

\title{ 2.4 The Execution of Commands}
\author{Wataru Yachi}
\institute{JAIST}
\date{June 15, 2023}

\begin{document}

\maketitle

\begin{frame}
    \frametitle{Execution of Commands}
    \begin{definition}[execution relation]
        A relation
        \[
            \langle c, \sigma \rangle \rightarrow \sigma'
        \]
        means that execution of command $c$ in state $\sigma$ terminates in final state $\sigma'$.
    \end{definition}


    \begin{example}
        $\langle X \coloneqq 5, \sigma \rangle \rightarrow \sigma'$
    \end{example}
\end{frame}

\begin{frame}
    \frametitle{Replacement of Location}

    \begin{block}{Notation}
        Let $\sigma$ be a state. Let $m \in \mathbf{N}$. Let $X \in \mathbf{Loc}$.
        A new state which obteined from $\sigma$ by replacing its contents in $X$ by $m$ is denoted as follows;
        \[ \sigma \lbrack m / X \rbrack. \]
        And we have
        \[
            \sigma \lbrack m/X \rbrack (Y) = \left\{ \begin{array}{lcl}
                m & \mathrm{if} & Y = X\\
                \sigma(Y) & \mathrm{if} & Y \neq X
            \end{array} \right.
        \]
    \end{block}

    \begin{example}[Quiz]
        \begin{itemize}
            \item $(\sigma[5/X])[3/Y](X) = \m{?}$
            \item $\sigma[2/Y](X) = \m{?}$
        \end{itemize}
    \end{example}
\end{frame}

\begin{frame}
    \frametitle{Rules for Commands 1}

    \begin{definition}
        Atomic commands
        \begin{align*}
            \infer{\langle \mathbf{skip} , \sigma\rangle \rightarrow \sigma}{} 
            \;\;\;\; 
            \infer{\cpair{X \coloneqq a, \sigma} \to \sigma[m/X]}{\cpair{a, \sigma} \to m}
        \end{align*}
        Sequencing
            \begin{align*}
            \infer{\langle c_0;c_1, \sigma \rangle \rightarrow \sigma'}{\langle c_0, \sigma \rangle \to \sigma'' & \langle c_1, \sigma'' \rangle \to \sigma'}
        \end{align*}
        Conditionals
        \begin{align*}\infer{\langle \ifthen{b}{c_0}{c_1}, \sigma \rangle \to \sigma'}
            {\cpair{b,\sigma} \to \mathbf{true} & \cpair{c_0, \sigma} \to \sigma'}
            \;\;\;\;\; \infer{\cpair{\ifthen{b}{c_0}{c_1}, \sigma} \to \sigma'}
                {\cpair{b,\sigma} \to \mathbf{false} & \cpair{c_1, \sigma} \to \sigma'}
        \end{align*}
    \end{definition}
\end{frame}

\begin{frame}
    \frametitle{Rules for Commands 2}
    \begin{definition}
        While-loops
        \begin{gather*}
                \infer{\cpair{\while{b}{c}, \sigma} \to \sigma}{\cpair{b, \sigma} \to \mathbf{false}}\\
                \infer{\cpair{\while{b}{c}, \sigma} \to \sigma'}
                {\cpair{b,\sigma} \to \mathbf{true} & \cpair{c, \sigma} \to \sigma'' & \cpair{\while{b}{c},\sigma''} \to \sigma'}
        \end{gather*}

    \end{definition}
\end{frame}

\begin{frame}
    \frametitle{Equivalence Relation on Commands}
    \begin{definition}[equivalence relation $\sim$ on commands]
        Let $c_0$ and $c_1$ be a command,  
        a equivalence relation $\sim$ on commands is defined as follows;
        \[
            c_0 \sim c_1 \coloneqq \forall \sigma, \sigma' \in \Sigma . \cpair{c_0, \sigma} \to \sigma' \iff \cpair{c_1, \sigma} \to \sigma'.
        \]
    \end{definition}

    \begin{example}[Quiz]
        \begin{itemize}
            \item $X \coloneqq 5 \sim  X \coloneqq 2 + 3$ ? 
            \item $\ifthen{\mathbf{true}}{X \coloneqq 5}{X \coloneqq 0} \sim \ifthen{\mathbf{false}}{X \coloneqq 0}{X \coloneqq 5}$ ?
            \item $X \coloneqq 5 ; \mathbf{skip} \sim X \coloneqq 5 ; \while{\mathbf{false}}{X \coloneqq 0}$ ?
        \end{itemize}
    \end{example}

\end{frame}
\end{document}
