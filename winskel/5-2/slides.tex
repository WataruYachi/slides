% 4up version
%\documentclass[handout, 12pt, aspectratio=169]{beamer}

\documentclass[12pt,aspectratio=169]{beamer}

\usefonttheme{professionalfonts}

\mode<handout>
{
\usepackage{pgfpages}
\pgfpagesuselayout{4 on 1}[a4paper,landscape]
}

\usepackage[english]{babel}
\usepackage{luatexja}
\usepackage{luatexja-fontspec}

\usepackage{autobreak}

% for display code
%\usepackage{minted}
%\newfontfamily\hasklig{hasklig}[NFSSFamily=haskligFamily]
%\setminted[haskell]{fontfamily=haskligFamily,
%mathescape,
%       numbersep=5pt
%}

% for proof tree
%\usepackage{bcprules}

%\usepackage[T1]{fontenc}

%for lualatex
\usepackage{fontspec}
\setsansfont{CMU Sans Serif}%{Arial}
\setmainfont{CMU Serif}%{Times New Roman}
\setmonofont{CMU Typewriter Text}%{Consolas}

\usepackage{lmodern,amsmath,amssymb,proof}
\usepackage{tikz}
\usetikzlibrary{arrows,positioning}

\usepackage{stmaryrd}



% color scheme
\definecolor{green}{rgb}{0.0,0.5,0.0}
\definecolor{blue}{rgb}{0.0,0.0,0.7}
\definecolor{red}{rgb}{0.8,0.0,0.0}
\definecolor{lightred}{rgb}{1.0,0.97,0.97}
\definecolor{lightblue}{rgb}{0.95,0.95,1.0}
\definecolor{darkblue}{rgb}{0.3,0.3,0.5}
\definecolor{darkred}{rgb}{0.6,0.0,0.0}
\definecolor{darkorange}{rgb}{0.6,0.2,0.1}
\definecolor{darkgreen}{rgb}{0,0.3,0}
%
\newcommand{\IMAGE}[2]{\pgfdeclareimage[#1]{#2}{#2}\pgfuseimage{#2}}

\setbeamerfont{title}{series=\bfseries,size=\Large}
\setbeamercolor{title}{fg=red}
\setbeamerfont{subtitle}{series=\bfseries,size=\LARGE}
\setbeamercolor{subtitle}{fg=red}
\setbeamerfont{author}{series=\bfseries,size=\large}
\setbeamercolor{author}{fg=blue}
\setbeamerfont{institute}{series=\bfseries,size=\large}
\setbeamercolor{institute}{fg=green}

\setbeamertemplate{blocks}[rounded]
\setbeamerfont{block title}{series=\bfseries,family=\sffamily,size=\small\strut}
\setbeamercolor{block title}{bg=darkblue,fg=white}
\setbeamercolor{block body}{bg=blue!4!white}
\setbeamercolor{block title example}{bg=white,fg=darkgreen}
\setbeamercolor{block body example}{bg=white}
\setbeamercolor{block title alerted}{bg=darkred,fg=white}
\setbeamercolor{block body alerted}{bg=red!4!white}
\setbeamercolor{structure}{fg=darkred}
% no fading effect
\makeatletter
\pgfdeclareverticalshading[lower.bg,upper.bg]{bmb@transition}{200cm}{%
  color(0pt)=(lower.bg); color(4pt)=(lower.bg); color(4pt)=(upper.bg)}
\makeatother

% frametitle
\useframetitletemplate{
\begin{centering}
\centerline{\large\bfseries\color{darkblue}\insertframetitle}
\end{centering}
}

% footer  (title  page/pages)
\setbeamertemplate{navigation symbols}{}
\useheadtemplate{\vbox{\vskip8pt}}
\usefoottemplate{\vbox{\vskip2pt\inserttitle\hfil\insertframenumber/\inserttotalframenumber\vskip5pt}}

% enumerate/itemize environment
\newcommand*\tikzboxed[1]{\tikz[baseline=(c.base)]{%
\node[thick,shape=rectangle,draw,inner sep=2pt] (c) {#1};}}
\setbeamertemplate{items}[square]
\setbeamertemplate{enumerate item}{\tikzboxed{\footnotesize\insertenumlabel}}
\setlength{\itemsep}{1ex}

\newcommand{\m}[1]{\mathsf{#1}}
\newcommand{\mi}[1]{\mathit{#1}}
\newcommand{\md}[1]{\mathtt{\textcolor{blue}{#1}}}
\newcommand{\seq}[2][n]{{#2_1},\dots,{#2_{#1}}}
%
\newcommand{\FF}{\mathcal{F}}
\newcommand{\RR}{\mathcal{R}}
\newcommand{\VV}{\mathcal{V}}
\newcommand{\TT}{\mathcal{T}}
\newcommand{\Var}{\mathcal{V}\m{ar}}
%
\newcommand{\app}{\circ}
\newcommand{\Aexp}{\mathbf{Aexp}}
\newcommand{\Bexp}{\mathbf{Bexp}}
\newcommand{\Com}{\mathbf{Com}}

\newcommand{\denoA}[1]{\mathcal{A} \llbracket #1 \rrbracket}
\newcommand{\denoB}[1]{\mathcal{B} \llbracket #1 \rrbracket}
\newcommand{\denoC}[1]{\mathcal{C} \llbracket #1 \rrbracket}

\newcommand{\true}{\mathbf{true}}
\newcommand{\false}{\mathbf{false}}
\newcommand{\Skip}{\mathbf{skip}}
\def\coloneqq{\mathrel{\mathop:}=}
\newcommand{\Assign}[2]{#1 \coloneqq #2}
\newcommand{\ITE}[3]{\mathbf{if}\; #1 \; \mathbf{then} \; #2 \; \mathbf{else} \; #3}
\newcommand{\While}[2]{\mathbf{while}\; #1 \; \mathbf{do} \; #2}
\newcommand{\deriv}[3]{\langle #1, #2 \rangle \rightarrow #3}


\title{5.2 Denotational Semantics}
\author{Wataru Yachi}
\institute{JAIST}
\date{October 5, 2023}

\begin{document}

\maketitle

\begin{frame}
    \frametitle{Semantic Functions}
    
    we will define semantic functions

    \begin{itemize}
        \item $\mathcal{A}:\Aexp \rightarrow (\Sigma \rightarrow \mathbf{N})$
        \item $\mathcal{B}:\Bexp \rightarrow (\Sigma \rightarrow \mathbf{T})$
        \item $\mathcal{C}:\Com \rightarrow (\Sigma \rightharpoonup \Sigma)$
    \end{itemize}

    \pause
    \bigskip

    for example,
    \begin{itemize}
        \item command $c$ is said to \alert{denote} $\denoC{c}$
        \item $\denoC{c}$ is said to be \alert{denotation} of $c$
    \end{itemize}
\end{frame}

\begin{frame}
    \frametitle{Denotations of $\Aexp$}
    denotations of $\Aexp$ are defined as functions from states to numbers
        \begin{align*}
            \denoA{n} & = \{(\sigma,n) \mid \sigma \in \Sigma \}\\
            \denoA{X} & = \{ (\sigma, \sigma(X)) \mid \sigma \in \Sigma \}\\
            \denoA{a_0 + a_1} & = \{(\sigma, n_0 + n_1) \mid
                (\sigma, n_0) \in \denoA{a_0} \; \& \; (\sigma, n_1) \in \denoA{a_1}\}\\
            \denoA{a_0 - a_1} & = \{(\sigma, n_0 - n_1) \mid
                (\sigma, n_0) \in \denoA{a_0} \; \& \; (\sigma, n_1) \in \denoA{a_1}\}\\
            \denoA{a_0 \times a_1} & = \{(\sigma, n_0 \times n_1) \mid
                (\sigma, n_0) \in \denoA{a_0} \; \& \; (\sigma, n_1) \in \denoA{a_1}\}
        \end{align*}
        \vspace{-10mm}
        \pause
    \begin{example}
        let $\sigma = \{(X,1),(Y,2)\}$
        \begin{itemize}[<+->]
            \item $\denoA{1} = 1$, \pause $\denoA{X}\sigma = \sigma(X) = 1$
            \item $\denoA{X \times (Y + 3)}\sigma = \denoA{X}\sigma \times (\denoA{Y + 3}\sigma) = 1 \times (2 + 3) = 5$
        \end{itemize}
    \end{example}
\end{frame}

\begin{frame}
    \frametitle{Denotations of $\Bexp$}
    denotations of $\Bexp$ are defined as functions from states to truth values
        \begin{align*}
            \denoB{\true} = & \; \{(\sigma, \true) \mid \sigma \in \Sigma\} \\
            \denoB{\false} = & \; \{ (\sigma,\false) \mid \sigma \in \Sigma\} \\
            \denoB{a_0 = a_1} = & \;
            \{ (\sigma, \true) \mid \sigma \in \Sigma \; \& \; \denoA{a_0}\sigma = \denoA{a_1}\sigma\} \; \cup\\
            & \; \{ (\sigma, \false) \mid \sigma \in \Sigma \; \& \; \denoA{a_0}\sigma \neq \denoA{a_1}\sigma\}\\
            \denoB{b_0 \land b_1} = & \; \{(\sigma, t_0 \land_T t_1) \mid \sigma \in \Sigma \; \& \; (\sigma,t_0) \in \denoB{b_0} \; \& \; (\sigma,t_1) \in \denoB{b_1} \}
        \end{align*}
        \vspace{-10mm}
        \pause
    \begin{example}
        let $\sigma = \{(X,1),(Y,2)\}$\\ \pause
        $\denoB{(X + 1) = Y}\sigma \pause = \true$, \pause because $\denoA{(X+1)}\sigma = \denoA{Y}\sigma$ 
    \end{example}
\end{frame}

\begin{frame}
    \frametitle{Denotations of $\Com$}
    denotations of $\Com$ are defined as functions from state to state
    \begin{align*}
        \denoC{\Skip} =& \;\{(\sigma,\sigma) \mid \sigma \in \Sigma\}\\
        \denoC{\Assign{X}{a}} =& \;\{(\sigma,\sigma[n/X]) \mid \sigma \in \Sigma \; \& \; n = \denoA{a}\sigma\}\\
        \denoC{c_0;c_1} =& \;\denoC{c_1} \circ \denoC{c_0}\\
        \denoC{\ITE{b}{c_0}{c_1}} =& \;
            \{(\sigma,\sigma') \mid \denoB{b}\sigma = \true \; \& \; (\sigma, \sigma') \in \denoC{c_0}\} \; \cup\\
            & \;\{(\sigma,\sigma') \mid \denoB{b}\sigma = \false \; \& \; (\sigma, \sigma') \in \denoC{c_1}\}
    \end{align*}
    \pause
    but how we define denotation of $\While{b}{c}$?
\end{frame}

\begin{frame}
    \frametitle{Denotation of While-Loop}
    let $w \equiv \While{b}{c}$
    \pause
    recall that
    \[
        w \sim \ITE{b}{c;w}{\Skip}
    \]
    \pause
    so we can assume that
    \begin{align*}
        \denoC{w} = & \; \denoC{\ITE{b}{c;w}{\Skip}}\\ \pause
            = & \; \{(\sigma, \sigma') \mid \denoB{b}\sigma = \true \; \& \; (\sigma, \sigma') \in \denoC{w} \circ \denoC{c}\} \; \cup \\
            & \; \{ (\sigma,\sigma) \mid \denoB{b}\sigma = \false \}
    \end{align*}
    \pause
    $\denoC{w}$ appears in both side of equation, how to solve it?
    %writing $\varphi$ for $\denoC{w}$, $\beta$ for $\denoB{b}$ and $\gamma$ for $\denoC{c}$
    
\end{frame}

\begin{frame}
    \frametitle{Fixed Point of $\Gamma$ Is Denotation of While-Loop}
    we assume $\Gamma$, where
        \begin{align*}
            \Gamma(\varphi) %= & \; \{(\sigma, \sigma') \mid \beta(\sigma) = \true \; \& \; (\sigma,\sigma') \in \varphi \circ \gamma\} \; \cup \\
                            %& \; \{(\sigma, \sigma) \mid \beta(\sigma) = \false\}\\
                            = & \; \{(\sigma, \sigma') \mid \exists \sigma'' .\, \beta(\sigma) = \true \; \& \; (\sigma, \sigma'') \in \gamma \; \& \; (\sigma'', \sigma') \in \varphi\} \; \cup \\
                            & \; \{(\sigma,\sigma) \mid \beta(\sigma) = \false \}
        \end{align*}
    $\Gamma$ is equal to $\widehat{R}$
    \begin{align*}
        \Gamma(\varphi) = \widehat{R} = \{(\sigma,\sigma') \mid \exists \{(\sigma'',\sigma')\} \subseteq \varphi \, . \, \{(\sigma'',\sigma')\} / (\sigma, \sigma') \in R \}
    \end{align*}
    and $\widehat{R}$ is determined by rule instances $R$
    \begin{align*}
        R = & \; \{ (\{(\sigma'', \sigma')\} / (\sigma, \sigma')) \mid \beta(\sigma) = \true \; \& \; (\sigma, \sigma'') \in \gamma\}\; \cup \\
        & \; \{(\emptyset / (\sigma,\sigma)) \mid \beta(\sigma) = \false \}
    \end{align*}

    and we know $\widehat{R}$ has least fixed point
\end{frame}



\begin{frame}
    %\frametitle{Denotations of $\Com$}
    finally, we have
    \begin{align*}
        \denoC{\Skip} =& \;\{(\sigma,\sigma) \mid \sigma \in \Sigma\}\\
        \denoC{\Assign{X}{a}} =& \;\{(\sigma,\sigma[n/X]) \mid \sigma \in \Sigma \; \& \; n = \denoA{a}\sigma\}\\
        \denoC{c_0;c_1} =& \;\denoC{c_1} \circ \denoC{c_0}\\
        \denoC{\ITE{b}{c_0}{c_1}} =& \;
            \{(\sigma,\sigma') \mid \denoB{b}\sigma = \true \; \& \; (\sigma, \sigma') \in \denoC{c_0}\} \; \cup\\
            & \;\{(\sigma,\sigma') \mid \denoB{b}\sigma = \false \; \& \; (\sigma, \sigma') \in \denoC{c_1}\}\\
        \denoC{\While{b}{c}} = & \; \mathrm{fix}(\Gamma)
    \end{align*}
    where
    \begin{align*}
        \Gamma(\varphi) = & \; \{ (\sigma, \sigma') \mid \denoB{b}\sigma = \true \; \& \; (\sigma, \sigma') \in \varphi \circ \denoC{c}\} \; \cup \\
        & \; \{(\sigma,\sigma) \mid \denoB{b}\sigma = \false \}
    \end{align*}
\end{frame}

\begin{frame}
    \begin{block}{Proposition}
        $\denoC{\While{\false}{c_0};c_1} = \denoC{c_1}$
    \end{block}
    \pause
    \begin{proof}
        \begin{align*}
            \action<+->{\denoC{\While{\false}{c_0};c_1} & = \denoC{c_1} \circ \denoC{\While{\false}{c_0}}\\}
            \action<+->{& = \denoC{c_1} \circ \Gamma(\denoC{\While{\false}{c_0}})\\}
            \action<+->{& = \denoC{c_1} \circ \{(\sigma,\sigma) \mid \sigma \in \Sigma\}\\}
            \action<+->{& = \denoC{c_1}}
        \end{align*}
    \end{proof}
\end{frame}

\begin{frame}
    \begin{block}{Claim (Exercise 5.5 (1))}
        for all arithmetic expressions $a$, $\{(\sigma, n) \mid \langle a, \sigma \rangle \rightarrow n\} \subseteq \denoA{a}$
    \end{block}
    \begin{proof}
        We show the claim by rule induction on arithmetic expressions.
        Let $n$ be a number, $\sigma$ state and $X$ location.
        Fix $A = \{(\sigma, n) \mid \deriv{a}{\sigma}{n}\}$.
        \pause
        \begin{itemize}[<+->]
            \item Assume $(\sigma, n) \in A$. So there exists $\deriv{n}{\sigma}{n}$.
                Clearly, we have $(\sigma,n) \in \denoA{n}$. %for any $n$ and $\sigma$.
            \item Similarly, assume $(\sigma, \sigma(X)) \in A$ and we have $(\sigma, \sigma(X)) \in \denoA{X}$.% for all $X$ and $\sigma$.
            \item Assume $\deriv{a_0}{\sigma}{n_0}$ and $\deriv{a_1}{\sigma}{n_1}$.
                By I.H., we have $(\sigma, n_0) \in \denoA{a_0}$ and $(\sigma,n_1) \in \denoA{a_1}$.
                Suppose $(\sigma, n_0 + n_1) \in A$. %So we can derive $\deriv{a_0 + a_1}{\sigma}{n_0 + n_1}$.
                Thus $(\sigma, n_0 + n_1) \in \denoA{a_0 + a_1} $ holds.\\
                The other cases can be shown by the similar way.
        \end{itemize}
    \end{proof}
\end{frame}

\begin{frame}
    \begin{block}{Claim (Exercise 5.5 (2))}
        for all arithmetic expressions $a$, $\denoA{a} \subseteq \{(\sigma,n) \mid \deriv{a}{\sigma}{n}\}$
    \end{block}
    \begin{proof}
        We show the claim by structural induction on arithmetic expressions $a$.
        \pause
        \begin{itemize}[<+->]
            \item If $a$ is a number $m$, assume $(\sigma, n) \in \denoA{m}$. Clearly we have $\deriv{m}{\sigma}{n}$
                and $n = m$. So $(\sigma,n) \in \{(\sigma, n) \mid \deriv{a}{\sigma}{n}\}$.
            \item If $a$ is a location $X$, assume $(\sigma, n) \in \denoA{X}$.
                Then we have $\deriv{X}{\sigma}{\sigma(X)}$ and $\sigma(X) = n$. So $(\sigma,n) \in \{(\sigma, n) \mid \deriv{a}{\sigma}{n}\}$.
            \item If $a$ is of the form $a_0 + a_1$, assume $(\sigma, n) \in \denoA{a_0 + a_1}$.
                    We have $(\sigma,n_0) \in \denoA{a_0}$ and $(\sigma, n_1) \in \denoA{a_1}$.
                    By I.H., we obtain $\deriv{a_0}{\sigma}{n_0}$ and $\deriv{a_1}{\sigma}{n_1}$.
                    Hence we have $\deriv{a_0 + a_1}{\sigma}{n}$.
                    Thus $(\sigma,n) \in \{(\sigma, n) \mid \deriv{a}{\sigma}{n}\}$.\\

                    The remaining cases follows the same way.
        \end{itemize}
    \end{proof}
\end{frame}

\end{document}
