% 4up version
%\documentclass[handout, 12pt, aspectratio=169]{beamer}

\documentclass[12pt,aspectratio=169]{beamer}

\usefonttheme{professionalfonts}

\mode<handout>
{
\usepackage{pgfpages}
\pgfpagesuselayout{4 on 1}[a4paper,landscape]
}

\usepackage[english]{babel}
\usepackage{luatexja}
\usepackage{luatexja-fontspec}

\usepackage{autobreak}

% for display code
%\usepackage{minted}
%\newfontfamily\hasklig{hasklig}[NFSSFamily=haskligFamily]
%\setminted[haskell]{fontfamily=haskligFamily,
%mathescape,
%       numbersep=5pt
%}

% for proof tree
%\usepackage{bcprules}

%\usepackage[T1]{fontenc}

%for lualatex
\usepackage{fontspec}
\setsansfont{CMU Sans Serif}%{Arial}
\setmainfont{CMU Serif}%{Times New Roman}
\setmonofont{CMU Typewriter Text}%{Consolas}

\usepackage{lmodern,amsmath,amssymb,proof}
\usepackage{tikz}
\usetikzlibrary{arrows,positioning}

\usepackage{stmaryrd}



% color scheme
\definecolor{green}{rgb}{0.0,0.5,0.0}
\definecolor{blue}{rgb}{0.0,0.0,0.7}
\definecolor{red}{rgb}{0.8,0.0,0.0}
\definecolor{lightred}{rgb}{1.0,0.97,0.97}
\definecolor{lightblue}{rgb}{0.95,0.95,1.0}
\definecolor{darkblue}{rgb}{0.3,0.3,0.5}
\definecolor{darkred}{rgb}{0.6,0.0,0.0}
\definecolor{darkorange}{rgb}{0.6,0.2,0.1}
\definecolor{darkgreen}{rgb}{0,0.3,0}
%
\newcommand{\IMAGE}[2]{\pgfdeclareimage[#1]{#2}{#2}\pgfuseimage{#2}}

\setbeamerfont{title}{series=\bfseries,size=\Large}
\setbeamercolor{title}{fg=red}
\setbeamerfont{subtitle}{series=\bfseries,size=\LARGE}
\setbeamercolor{subtitle}{fg=red}
\setbeamerfont{author}{series=\bfseries,size=\large}
\setbeamercolor{author}{fg=blue}
\setbeamerfont{institute}{series=\bfseries,size=\large}
\setbeamercolor{institute}{fg=green}

\setbeamertemplate{blocks}[rounded]
\setbeamerfont{block title}{series=\bfseries,family=\sffamily,size=\small\strut}
\setbeamercolor{block title}{bg=darkblue,fg=white}
\setbeamercolor{block body}{bg=blue!4!white}
\setbeamercolor{block title example}{bg=white,fg=darkgreen}
\setbeamercolor{block body example}{bg=white}
\setbeamercolor{block title alerted}{bg=darkred,fg=white}
\setbeamercolor{block body alerted}{bg=red!4!white}
\setbeamercolor{structure}{fg=darkred}
% no fading effect
\makeatletter
\pgfdeclareverticalshading[lower.bg,upper.bg]{bmb@transition}{200cm}{%
  color(0pt)=(lower.bg); color(4pt)=(lower.bg); color(4pt)=(upper.bg)}
\makeatother

% frametitle
\useframetitletemplate{
\begin{centering}
\centerline{\large\bfseries\color{darkblue}\insertframetitle}
\end{centering}
}

% footer  (title  page/pages)
\setbeamertemplate{navigation symbols}{}
\useheadtemplate{\vbox{\vskip8pt}}
\usefoottemplate{\vbox{\vskip2pt\inserttitle\hfil\insertframenumber/\inserttotalframenumber\vskip5pt}}

% enumerate/itemize environment
\newcommand*\tikzboxed[1]{\tikz[baseline=(c.base)]{%
\node[thick,shape=rectangle,draw,inner sep=2pt] (c) {#1};}}
\setbeamertemplate{items}[square]
\setbeamertemplate{enumerate item}{\tikzboxed{\footnotesize\insertenumlabel}}
\setlength{\itemsep}{1ex}

\newcommand{\m}[1]{\mathsf{#1}}
\newcommand{\mi}[1]{\mathit{#1}}
\newcommand{\md}[1]{\mathtt{\textcolor{blue}{#1}}}
\newcommand{\seq}[2][n]{{#2_1},\dots,{#2_{#1}}}
%
\newcommand{\FF}{\mathcal{F}}
\newcommand{\RR}{\mathcal{R}}
\newcommand{\VV}{\mathcal{V}}
\newcommand{\TT}{\mathcal{T}}
\newcommand{\Var}{\mathcal{V}\m{ar}}
%
\newcommand{\app}{\circ}
\newcommand{\Aexp}{\mathbf{Aexp}}
\newcommand{\Bexp}{\mathbf{Bexp}}
\newcommand{\Com}{\mathbf{Com}}

\newcommand{\denoA}[1]{\mathcal{A} \llbracket #1 \rrbracket}
\newcommand{\denoB}[1]{\mathcal{B} \llbracket #1 \rrbracket}
\newcommand{\denoC}[1]{\mathcal{C} \llbracket #1 \rrbracket}

\newcommand{\true}{\mathbf{true}}
\newcommand{\false}{\mathbf{false}}
\newcommand{\Skip}{\mathbf{skip}}
\def\coloneqq{\mathrel{\mathop:}=}
\newcommand{\Assign}[2]{#1 \coloneqq #2}
\newcommand{\ITE}[3]{\mathbf{if}\; #1 \; \mathbf{then} \; #2 \; \mathbf{else} \; #3}
\newcommand{\While}[2]{\mathbf{while}\; #1 \; \mathbf{do} \; #2}


\title{5.2 Denotational Semantics}
\author{Wataru Yachi}
\institute{JAIST}
\date{October 5, 2023}

\begin{document}

\maketitle

\begin{frame}
    \frametitle{Denotations of $\Aexp$}
    \begin{definition}
        \begin{align*}
            & \denoA{n} = \{(\sigma,n) \mid \sigma \in \Sigma \}\\
            & \denoA{X} = \{ (\sigma, \sigma(X)) \mid \sigma \in \Sigma \}\\
            & \denoA{a_0 + a_1} = \{(\sigma, n_0 + n_1) \mid
                (\sigma, n_0) \in \denoA{a_0} \; \& \; (\sigma, n_1) \in \denoA{a_1}\}\\
            & \denoA{a_0 - a_1} = \{(\sigma, n_0 - n_1) \mid
                (\sigma, n_0) \in \denoA{a_0} \; \& \; (\sigma, n_1) \in \denoA{a_1}\}\\
            & \denoA{a_0 \times a_1} = \{(\sigma, n_0 \times n_1) \mid
                (\sigma, n_0) \in \denoA{a_0} \; \& \; (\sigma, n_1) \in \denoA{a_1}\}\\
        \end{align*}
    \end{definition}
    \begin{example}
        
    \end{example}
\end{frame}

\begin{frame}
    \frametitle{Denotations of $\Bexp$}
    \begin{definition}
        \begin{align*}
            & \denoB{\true} = \{(\sigma, \true) \mid \sigma \in \Sigma\} \\
            & \denoB{\false} = \{ (\sigma,\false) \mid \sigma \in \Sigma\} \\
            & \denoB{a_0 = a_1} =
            \{ (\sigma, \true) \mid \sigma \in \Sigma \; \& \; \denoA{a_0}\sigma = \denoA{a_1}\sigma\}\cup\\
            & \{ (\sigma, \false) \mid \sigma \in \Sigma \; \& \; \denoA{a_0}\sigma \neq \denoA{a_1}\sigma\}\\
            & \denoB{b_0 \land b_1} = \{ \mid \sigma \in \Sigma \; \& \; (\sigma,t_1) \in \denoB{b_0} \; \& \; (\sigma,b_1) \in \denoB{b_1} \}
        \end{align*}
    \end{definition}
\end{frame}
\begin{example}

\end{example}

\begin{frame}
    \frametitle{Denotations of $\Com$}
    denotations of $\Com$ are defined as function from state to state
    \begin{align*}
        \denoC{\Skip} =& \;\{(\sigma,\sigma) \mid \sigma \in \Sigma\}\\
        \denoC{\Assign{X}{a}} =& \;\{(\sigma,\sigma[n/X]) \mid \sigma \in \Sigma \; \& \; n = \denoA{a}\sigma\}\\
        \denoC{c_0;c_1} =& \;\denoC{c_1} \circ \denoC{c_0}\\
        \denoC{\ITE{b}{c_0}{c_1}} =& \;
            \{(\sigma,\sigma') \mid \denoB{b}\sigma = \true \; \& \; (\sigma, \sigma') \in \denoC{c_0}\} \cup\\
            & \;\{(\sigma,\sigma') \mid \denoB{b}\sigma = \false \; \& \; (\sigma, \sigma') \in \denoC{c_1}\}
    \end{align*}
    but how we define $\While{b}{c}$?
\end{frame}

\begin{frame}
    \frametitle{Denotation of While-Loop}
    let $w \equiv \While{b}{c}$
    recall that 
\end{frame}

\end{document}
