% 4up version
%\documentclass[handout, 12pt, aspectratio=169]{beamer}

\documentclass[12pt,aspectratio=169]{beamer}

\usefonttheme{professionalfonts}

\mode<handout>
{
\usepackage{pgfpages}
\pgfpagesuselayout{4 on 1}[a4paper,landscape]
}

\usepackage[english]{babel}
\usepackage{luatexja}
\usepackage{luatexja-fontspec}


% for display code
%\usepackage{minted}
%\newfontfamily\hasklig{hasklig}[NFSSFamily=haskligFamily]
%\setminted[haskell]{fontfamily=haskligFamily,
%mathescape,
%       numbersep=5pt
%}

% for proof tree
%\usepackage{bcprules}

%\usepackage[T1]{fontenc}

%for lualatex
\usepackage{fontspec}
\setsansfont{CMU Sans Serif}%{Arial}
\setmainfont{CMU Serif}%{Times New Roman}
\setmonofont{CMU Typewriter Text}%{Consolas}

\usepackage{lmodern,amsmath,amssymb,proof}
\usepackage{tikz}
\usetikzlibrary{arrows,positioning}



% color scheme
\definecolor{green}{rgb}{0.0,0.5,0.0}
\definecolor{blue}{rgb}{0.0,0.0,0.7}
\definecolor{red}{rgb}{0.8,0.0,0.0}
\definecolor{lightred}{rgb}{1.0,0.97,0.97}
\definecolor{lightblue}{rgb}{0.95,0.95,1.0}
\definecolor{darkblue}{rgb}{0.3,0.3,0.5}
\definecolor{darkred}{rgb}{0.6,0.0,0.0}
\definecolor{darkorange}{rgb}{0.6,0.2,0.1}
\definecolor{darkgreen}{rgb}{0,0.3,0}
%
\newcommand{\IMAGE}[2]{\pgfdeclareimage[#1]{#2}{#2}\pgfuseimage{#2}}

\setbeamerfont{title}{series=\bfseries,size=\Large}
\setbeamercolor{title}{fg=red}
\setbeamerfont{subtitle}{series=\bfseries,size=\LARGE}
\setbeamercolor{subtitle}{fg=red}
\setbeamerfont{author}{series=\bfseries,size=\large}
\setbeamercolor{author}{fg=blue}
\setbeamerfont{institute}{series=\bfseries,size=\large}
\setbeamercolor{institute}{fg=green}

\setbeamertemplate{blocks}[rounded]
\setbeamerfont{block title}{series=\bfseries,family=\sffamily,size=\small\strut}
\setbeamercolor{block title}{bg=darkblue,fg=white}
\setbeamercolor{block body}{bg=blue!4!white}
\setbeamercolor{block title example}{bg=white,fg=darkgreen}
\setbeamercolor{block body example}{bg=white}
\setbeamercolor{block title alerted}{bg=darkred,fg=white}
\setbeamercolor{block body alerted}{bg=red!4!white}
\setbeamercolor{structure}{fg=darkred}
% no fading effect
\makeatletter
\pgfdeclareverticalshading[lower.bg,upper.bg]{bmb@transition}{200cm}{%
  color(0pt)=(lower.bg); color(4pt)=(lower.bg); color(4pt)=(upper.bg)}
\makeatother

% frametitle
\useframetitletemplate{
\begin{centering}
\centerline{\large\bfseries\color{darkblue}\insertframetitle}
\end{centering}
}

% footer  (title  page/pages)
\setbeamertemplate{navigation symbols}{}
\useheadtemplate{\vbox{\vskip8pt}}
\usefoottemplate{\vbox{\vskip2pt\inserttitle\hfil\insertframenumber/\inserttotalframenumber\vskip5pt}}

% enumerate/itemize environment
\newcommand*\tikzboxed[1]{\tikz[baseline=(c.base)]{%
\node[thick,shape=rectangle,draw,inner sep=2pt] (c) {#1};}}
\setbeamertemplate{items}[square]
\setbeamertemplate{enumerate item}{\tikzboxed{\footnotesize\insertenumlabel}}
\setlength{\itemsep}{1ex}

\newcommand{\m}[1]{\mathsf{#1}}
\newcommand{\mi}[1]{\mathit{#1}}
\newcommand{\md}[1]{\mathtt{\textcolor{blue}{#1}}}
\newcommand{\seq}[2][n]{{#2_1},\dots,{#2_{#1}}}
%
\newcommand{\FF}{\mathcal{F}}
\newcommand{\RR}{\mathcal{R}}
\newcommand{\VV}{\mathcal{V}}
\newcommand{\TT}{\mathcal{T}}
\newcommand{\Var}{\mathcal{V}\m{ar}}
%
\newcommand{\app}{\circ}

\title{ Exercise 5.10 and 5.14 }
\author{Wataru Yachi}
\institute{JAIST}
\date{month DD, YYYY}

\begin{document}

\begin{frame}
    \frametitle{Exercise 5.10 (1)}
    \begin{block}{Claim}
        for all set $X$, $(Pow(X), \subseteq)$ is cpo with bottom
    \end{block}
    \pause
    \begin{proof}
        For all set X, $(Pow(X), \subseteq)$ is a partial order because the relation $\subseteq$ is
        reflexive, transitive, and antisymmetric.
        \pause
        Obviously, $\emptyset$ is a least element.
        \pause
        Let $\{X_n \mid n \in \omega \}$ be a increasing chain
        $X_0 \subseteq X_1 \subseteq \dots \subseteq X_n \subseteq \cdots $ of subsets of $X$.
        \pause
        Take $C = \bigcup \{X_n \mid n \in \omega \}$.
        \pause
        So for all $X_n \in \{X_n \mid n \in \omega\}$, $X_n \subseteq C$ and
        for every upper bound $U$ of $\{X_n \mid n \in \omega\}$, $C \subseteq U$ holds.
        \pause
        Thus $C$ is a least upper bound of $\{X_n \mid n \in \omega \}$ in $Pow(X)$.
        \pause
        Hence $(Pow(X), \subseteq)$ is a cpo with bottom.
    \end{proof}
\end{frame}

\begin{frame}
    \frametitle{Exercise 5.10 (2)}
    \begin{block}{Claim}
        set of partial functions $\Sigma \rightharpoonup \Sigma$ ordered by $\subseteq$ is cpo with bottom
    \end{block}
    \pause
    \begin{proof}
        A partial functions $\Sigma \rightharpoonup \Sigma$ are subsets of $\Sigma \times \Sigma$.
        So according to the previous question, the claim holds.
    \end{proof}
\end{frame}

\begin{frame}
    \frametitle{Exercise 5.14}
    \begin{block}{Claim}
        let $(L, \sqsubseteq)$ be complete lattice, every subset of $L$ has least upper bound
    \end{block}
    \pause

    \begin{proof}
        Let $S$ be a subset of $L$. \pause
        Take $U = \{u \in L \mid \forall s \in S . \, s \sqsubseteq u\} $.
        \pause
        So there exists some $p$ such that $p \sqsubseteq u$ for all $u \in U$.
        \pause
        By definition of glb, we have $s \sqsubseteq p$ for all $s \in S$.
        \pause
        Hence $p$ is a least upper bound of $S$.
    \end{proof}
\end{frame}

\end{document}
