% 4up version
%\documentclass[handout, 12pt, aspectratio=169]{beamer}

\documentclass[12pt,aspectratio=169]{beamer}

\usefonttheme{professionalfonts}

\mode<handout>
{
\usepackage{pgfpages}
\pgfpagesuselayout{4 on 1}[a4paper,landscape]
}

\usepackage[english]{babel}
\usepackage{luatexja}
\usepackage{luatexja-fontspec}


% for display code
%\usepackage{minted}
%\newfontfamily\hasklig{hasklig}[NFSSFamily=haskligFamily]
%\setminted[haskell]{fontfamily=haskligFamily,
%mathescape,
%       numbersep=5pt
%}

% for proof tree
%\usepackage{bcprules}

%\usepackage[T1]{fontenc}

%for lualatex
\usepackage{fontspec}
\setsansfont{CMU Sans Serif}%{Arial}
\setmainfont{CMU Serif}%{Times New Roman}
\setmonofont{CMU Typewriter Text}%{Consolas}

\usepackage{lmodern,amsmath,amssymb,proof}
\usepackage{tikz}
\usetikzlibrary{arrows,positioning}



% color scheme
\definecolor{green}{rgb}{0.0,0.5,0.0}
\definecolor{blue}{rgb}{0.0,0.0,0.7}
\definecolor{red}{rgb}{0.8,0.0,0.0}
\definecolor{lightred}{rgb}{1.0,0.97,0.97}
\definecolor{lightblue}{rgb}{0.95,0.95,1.0}
\definecolor{darkblue}{rgb}{0.3,0.3,0.5}
\definecolor{darkred}{rgb}{0.6,0.0,0.0}
\definecolor{darkorange}{rgb}{0.6,0.2,0.1}
\definecolor{darkgreen}{rgb}{0,0.3,0}
%
\newcommand{\IMAGE}[2]{\pgfdeclareimage[#1]{#2}{#2}\pgfuseimage{#2}}

\setbeamerfont{title}{series=\bfseries,size=\Large}
\setbeamercolor{title}{fg=red}
\setbeamerfont{subtitle}{series=\bfseries,size=\LARGE}
\setbeamercolor{subtitle}{fg=red}
\setbeamerfont{author}{series=\bfseries,size=\large}
\setbeamercolor{author}{fg=blue}
\setbeamerfont{institute}{series=\bfseries,size=\large}
\setbeamercolor{institute}{fg=green}

\setbeamertemplate{blocks}[rounded]
\setbeamerfont{block title}{series=\bfseries,family=\sffamily,size=\small\strut}
\setbeamercolor{block title}{bg=darkblue,fg=white}
\setbeamercolor{block body}{bg=blue!4!white}
\setbeamercolor{block title example}{bg=white,fg=darkgreen}
\setbeamercolor{block body example}{bg=white}
\setbeamercolor{block title alerted}{bg=darkred,fg=white}
\setbeamercolor{block body alerted}{bg=red!4!white}
\setbeamercolor{structure}{fg=darkred}
% no fading effect
\makeatletter
\pgfdeclareverticalshading[lower.bg,upper.bg]{bmb@transition}{200cm}{%
  color(0pt)=(lower.bg); color(4pt)=(lower.bg); color(4pt)=(upper.bg)}
\makeatother

% frametitle
\useframetitletemplate{
\begin{centering}
\centerline{\large\bfseries\color{darkblue}\insertframetitle}
\end{centering}
}

% footer  (title  page/pages)
\setbeamertemplate{navigation symbols}{}
\useheadtemplate{\vbox{\vskip8pt}}
\usefoottemplate{\vbox{\vskip2pt\inserttitle\hfil\insertframenumber/\inserttotalframenumber\vskip5pt}}

% enumerate/itemize environment
\newcommand*\tikzboxed[1]{\tikz[baseline=(c.base)]{%
\node[thick,shape=rectangle,draw,inner sep=2pt] (c) {#1};}}
\setbeamertemplate{items}[square]
\setbeamertemplate{enumerate item}{\tikzboxed{\footnotesize\insertenumlabel}}
\setlength{\itemsep}{1ex}

\newcommand{\m}[1]{\mathsf{#1}}
\newcommand{\mi}[1]{\mathit{#1}}
\newcommand{\md}[1]{\mathtt{\textcolor{blue}{#1}}}
\newcommand{\seq}[2][n]{{#2_1},\dots,{#2_{#1}}}
%
\newcommand{\FF}{\mathcal{F}}
\newcommand{\RR}{\mathcal{R}}
\newcommand{\VV}{\mathcal{V}}
\newcommand{\TT}{\mathcal{T}}
\newcommand{\Var}{\mathcal{V}\m{ar}}
%
\newcommand{\app}{\circ}

\newcommand{\cmd}[3]{\langle #1, #2 \rangle \to #3}
\newcommand{\cmdd}[2]

\title{3.2 Structural Induction}
\author{Wataru Yachi}
\institute{JAIST}
\date{June 22, 2023}

\begin{document}

\maketitle

\begin{frame}
    \frametitle{Structural Induction on Arithmetic Expressions}
    To prove some property $P$ for all arithmetic expressions $a$, we need to show:
    \begin{itemize}
        \item [1] for all $m \in \mathbf{N}$ $P(m)$ holds,
        \item [2] for all $X \in \mathbf{Loc}$ $P(X)$ holds and
        \item [3] for all $a_0, a_1 \in \mathbf{Aexp}$ $P(a_0)\; \& \; P(a_1)$ implise $P(a_0 * a_1)$ \\
            where $* \in \{+, -, \times\}$.
    \end{itemize}
    Sometimes we must not show the property holds for subexpressions ($a_0, a_1$ in the above).
    In that case, we can use a degenerate form of structural induction:
    \begin{itemize}
        \item [3] for all $a_0, a_1 \in \mathbf{Aexp}$ $P(a_0 * a_1)$ where $* \in \{+, -, \times\}$.
    \end{itemize}

\end{frame}

\begin{frame}
    \begin{block}{Claim}
        For all arithmetic expressions $a$, states $\sigma$
        and numbers $m, m'$
        \[
            \langle a, \sigma \rangle \to m \; \& \;
                \langle a, \sigma \rangle \to m'
                    \Rightarrow m = m'
        \]
    \end{block}

    \begin{Proof}
        We show the claim by structural induction on arithmetic expressions $a$.

        \begin{itemize}
            \item If $a$ is number $n$, $\cmd{a}{\sigma}{m}$ and $\cmd{a}{\sigma}{m'}$ implies $m = n = m'$.
            \item If $a = a_0 + a_1$, we have $\cmd{a_0}{\sigma}{m_0}$ and $\cmd{a_1}{\sigma}{m_1}$ with $ m = m_0 + m_1$
                and also, we have $\cmd{a_0}{\sigma}{m'_0}$ and $\cmd{a_1}{\sigma}{m'_1}$ with $ m' = m'_0 + m'_1$.\\ 
                By induction hypothesis $m_0 = m_0'$ and $m_1 = m'_1$.\\
                So $ m = m_0 + m_1 = m'_0 + m'_1 = m'$.
        \end{itemize}
        The remaning cases follow in a similar way.
    \end{Proof}
\end{frame}

\end{document}
