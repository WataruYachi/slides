% 4up version
%\documentclass[handout, 12pt, aspectratio=169]{beamer}

\documentclass[12pt,aspectratio=169]{beamer}

\usefonttheme{professionalfonts}

\mode<handout>
{
\usepackage{pgfpages}
\pgfpagesuselayout{4 on 1}[a4paper,landscape]
}

\usepackage[english]{babel}
\usepackage{luatexja}
\usepackage{luatexja-fontspec}


% for display code
%\usepackage{minted}
%\newfontfamily\hasklig{hasklig}[NFSSFamily=haskligFamily]
%\setminted[haskell]{fontfamily=haskligFamily,
%mathescape,
%       numbersep=5pt
%}

% for proof tree
%\usepackage{bcprules}

%\usepackage[T1]{fontenc}

%for lualatex
\usepackage{fontspec}
\setsansfont{CMU Sans Serif}%{Arial}
\setmainfont{CMU Serif}%{Times New Roman}
\setmonofont{CMU Typewriter Text}%{Consolas}

\usepackage{lmodern,amsmath,amssymb,proof}
\usepackage{tikz}
\usetikzlibrary{arrows,positioning}
\usepackage{stmaryrd}

\usepackage{comment}
\usepackage{listings}

% color scheme
\definecolor{green}{rgb}{0.0,0.5,0.0}
\definecolor{blue}{rgb}{0.0,0.0,0.7}
\definecolor{red}{rgb}{0.8,0.0,0.0}
\definecolor{lightred}{rgb}{1.0,0.97,0.97}
\definecolor{lightblue}{rgb}{0.95,0.95,1.0}
\definecolor{darkblue}{rgb}{0.3,0.3,0.5}
\definecolor{darkred}{rgb}{0.6,0.0,0.0}
\definecolor{darkorange}{rgb}{0.6,0.2,0.1}
\definecolor{darkgreen}{rgb}{0,0.3,0}
%
\newcommand{\IMAGE}[2]{\pgfdeclareimage[#1]{#2}{#2}\pgfuseimage{#2}}

\setbeamerfont{title}{series=\bfseries,size=\Large}
\setbeamercolor{title}{fg=red}
\setbeamerfont{subtitle}{series=\bfseries,size=\LARGE}
\setbeamercolor{subtitle}{fg=red}
\setbeamerfont{author}{series=\bfseries,size=\large}
\setbeamercolor{author}{fg=blue}
\setbeamerfont{institute}{series=\bfseries,size=\large}
\setbeamercolor{institute}{fg=green}

\setbeamertemplate{blocks}[rounded]
\setbeamerfont{block title}{series=\bfseries,family=\sffamily,size=\small\strut}
\setbeamercolor{block title}{bg=darkblue,fg=white}
\setbeamercolor{block body}{bg=blue!4!white}
\setbeamercolor{block title example}{bg=white,fg=darkgreen}
\setbeamercolor{block body example}{bg=white}
\setbeamercolor{block title alerted}{bg=darkred,fg=white}
\setbeamercolor{block body alerted}{bg=red!4!white}
\setbeamercolor{structure}{fg=darkred}
% no fading effect
\makeatletter
\pgfdeclareverticalshading[lower.bg,upper.bg]{bmb@transition}{200cm}{%
  color(0pt)=(lower.bg); color(4pt)=(lower.bg); color(4pt)=(upper.bg)}
\makeatother

% frametitle
\useframetitletemplate{
\begin{centering}
\centerline{\large\bfseries\color{darkblue}\insertframetitle}
\end{centering}
}

% footer  (title  page/pages)
\setbeamertemplate{navigation symbols}{}
\useheadtemplate{\vbox{\vskip8pt}}
\usefoottemplate{\vbox{\vskip2pt\inserttitle\hfil\insertframenumber/\inserttotalframenumber\vskip5pt}}

% enumerate/itemize environment
\newcommand*\tikzboxed[1]{\tikz[baseline=(c.base)]{%
\node[thick,shape=rectangle,draw,inner sep=2pt] (c) {#1};}}
\setbeamertemplate{items}[square]
\setbeamertemplate{enumerate item}{\tikzboxed{\footnotesize\insertenumlabel}}
\setlength{\itemsep}{1ex}

\newcommand{\m}[1]{\mathsf{#1}}
\newcommand{\mi}[1]{\mathit{#1}}
\newcommand{\md}[1]{\mathtt{\textcolor{blue}{#1}}}
\newcommand{\seq}[2][n]{{#2_1},\dots,{#2_{#1}}}
%
\newcommand{\FF}{\mathcal{F}}
\newcommand{\RR}{\mathcal{R}}
\newcommand{\VV}{\mathcal{V}}
\newcommand{\TT}{\mathcal{T}}
\newcommand{\Var}{\mathcal{V}\mathrm{ar}}
\newcommand{\LVar}{\mathcal{LV}\mathrm{ar}}
\newcommand{\II}{\mathcal{I}}
\newcommand{\JJ}{\mathcal{J}}
%
\newcommand{\app}{\circ}

\newcommand{\torule}{\to_{\mathtt{rule}}}
\newcommand{\tocalc}{\to_{\mathtt{calc}}}

\newcommand{\Terms}{\TT\mathrm{erms}}
\newcommand{\Stheory}{\Sigma_\mi{theory}}
\newcommand{\Sterms}{\Sigma_\mi{terms}}
\newcommand{\interpret}[1]{\llbracket #1 \rrbracket_\JJ}

\lstset{
  %language=c,
  basicstyle={\ttfamily\normalsize},
  %basicstyle=\normalsize,
  commentstyle=\textit,
  classoffset=1,
  keywordstyle=\bfseries,
  %frame=tRBl,
  %framesep=10pt,
  showstringspaces=false,
  %numbers=left,
  stepnumber=1,
  numberstyle=\footnotesize,
  tabsize=2,
  breaklines=true
}

\title{ Logically Constrained Term Rewriting Systems }
\author{Wataru Yachi}
\institute{JAIST}
\date{July 21, 2023}

\begin{document}

\maketitle

\begin{comment}
\begin{frame}
    \frametitle{If Treat Integers in Rules}

    \begin{example}
        for every $n,m,k \in \mathbb{Z}$
        \begin{tabular}{lcl}
            $\m{add}(\m{m}, \m{n}) \to k$ & if & $k = m + n$ \\
            $\m{even}(\m{n}) \to \m{true}$  & if & $n$ is even number\\
            $\m{even}(\m{n}) \to \m{false}$ & if & $n$ is not even number
            \end{tabular}
    \end{example}
\end{frame}
\end{comment}

\begin{frame}[fragile]
    \frametitle{Motivation}

\begin{center}
\begin{tabular}{c}
\begin{lstlisting}
function sum(int n){
    if (even(n))
        n = n;
    else
        n = n - 1;
    int x = 0;
    while(0 <= n) {
        x = x + n;
        n = n - 2;
    }
}
\end{lstlisting}
\end{tabular}
\end{center}
    \[
        \m{sum}(n) = \sum \{ x \mid 0 \leq x \leq n, \text{$x$ is even number}\}
    \]
\end{frame}

\begin{frame}
        TRS
        \vspace{2mm}
        {\small
        \begin{tabular}{ll}
        $\m{sum}(x) \to \m{sum2}(\m{even}(x),x)$ & \\
        $\m{sum2}(\m{true}, x) \to \m{sum3}(x)$ & $\m{sum2}(\m{false}, x)$ \to $\m{sum3}(\m{p}(x))$\\
        $\m{sum3}(\m{s}(\m{s}(x))) \to \m{add}(\m{s}(\m{s}(x)), \m{sum3}(x))$ & $\m{sum3}(\m{0}) \to \m{0}$\\
        $\m{even}(\m{s}(\m{s}(x))) \to \m{even}(x)$ & $\m{even}(\m{s}(x)) \to \m{false}$\\
        $\m{even}(\m{0}) \to \m{true}$ & \\
        $\m{p}(\m{s}(x)) \to x$ & $\m{p}(\m{0}) \to \m{0}$\\
        $\m{add}(x,\m{s}(y)) \to \m{s}(\m{add}(x,y))$ & $\m{add}(x,\m{0}) \to x$
        \end{tabular}
        }

        LCTRS
        \vspace{2mm}
        {\small
        \begin{tabular}{ll}
            $\m{sum}(x) \to \m{sum2}(x)$ & \\
            $\m{sum2}(x) \to \m{sum3}(x) \; [\m{even}(x)]$ & $\m{sum2}(x) \to \m{sum3}(x-1) \; [\neg(\m{even}(x))]$ \\
            $\m{sum3}(x) \to x + \m{sum3}(x) \; [\neg(0 \geq x)]$ & $\m{sum3}(x) \to \m{0} \; [0 \geq x]$
        \end{tabular}
        }
\end{frame}

\begin{frame}
    \frametitle{Signature}
    \begin{definition}[signature]
        \begin{itemize}
        \item \textbf{signature} $\Sigma$ is set of function symbols $f$
        \item each $f$ has \textbf{sort declaration}
        $[\iota_1 \times \cdots \times \iota_n] \Rightarrow \kappa$
        where all $\iota_i$ and $\kappa$ are sorts
        \item abbreviate $f : [] \Rightarrow \kappa$ to $f : \kappa$
    \end{itemize}
    \end{definition}
    \begin{example}
    \begin{tabular}{llll}
        $\m{0} : \m{int}$ & $\m{s} : [\m{int}] \Rightarrow \m{int}$ & $\m{true} : \m{bool}$ & $\m{false} : \m{bool}$ \\
        $\m{even} : [\m{int}] \Rightarrow \m{bool}$ &
        $\m{sum} : [\m{int}] \Rightarrow \m{int}$ & $\m{sum2} : [\m{bool} \times \m{int}] \Rightarrow \m{int}$ &
    \end{tabular}
    \end{example}
\end{frame}

\begin{frame}
    \frametitle{Term}
    \begin{definition}[term]
        \begin{itemize}
        \item let $\VV$ be set of variables each equipped with sort
        and $\Sigma$ be signature,
        \textbf{term} $s : \iota$ over $\Sigma$ and $\VV$ iff
        $\vdash s : \iota$ can be derived using following inference rules:
        \begin{align*}
            \infer{\vdash x : \iota}{x : \iota \in \VV} \;\;\;\;
                & \infer{\vdash f(s_1,\cdots, s_n) : \kappa}
                    {\vdash s_1 : \iota_1 & \cdots & \vdash s_n : \iota_n
                    & f:[\iota_1 \times \cdots \times \iota_n]
                    \Rightarrow \kappa \in \Sigma}
        \end{align*}
        \item in this time, we say term $s : \iota$ is \textbf{well-sorted}
        \item $\Terms(\Sigma, \VV)$ is set of all terms over $\Sigma$ and $\VV$
        \item $\Var(s)$ is set of all variables in $s$
        \item term $s$ is \textbf{ground} if $\Var(s) = \emptyset$
        \end{itemize}
    \end{definition}
\end{frame}


\begin{frame}
    \begin{example}
        \begin{itemize}
        \item $\m{sum2}(\m{even}(x), x) : \m{int}$ is well-sorted
        {\small
            \begin{align*}
                \infer{\vdash \m{sum2}(\m{even}(x), \m{0}) : \m{int}}
                {\infer{\vdash \m{even}(x) : \m{bool}}
                    {\infer{\vdash x : \m{int}}{x : \m{int} \in \VV} & \m{even} : [\m{int}] \Rightarrow \m{bool} \in \Sigma}
                & \infer{\vdash x : \m{int}}{x : \m{int} \in \VV}
                & \m{sum2} : [\m{bool} \times \m{int}] \Rightarrow \m{int} \in \Sigma
                }
            \end{align*}
        }
        \item $\m{sum2}(\m{s}(\m{0}), \m{false}) : \m{int}$ is ill-sorted
        \item $\m{sum}(\m{s}(\m{0}))$ is ground term
        \item $\m{sum}(x)$ is non-ground term
        \end{itemize}
    \end{example}
\end{frame}

\begin{frame}
    \frametitle{Substitution and Context}
    \begin{definition}[substitution]
        \begin{itemize}
            \item \textbf{substitution} $\gamma$ is mapping from variables to terms
        $\{x_1:\iota_1 \mapsto s_1 : \iota_1, \cdots x_n:\iota_n \mapsto s_n : \iota_n\}$
            \item $s\gamma$ denotes applying $\gamma$ to $s$
        \end{itemize}
    \end{definition}
    \begin{definition}[context]
        \begin{itemize}
            \item \textbf{context} $C$ is term which contains special variables $\Box_i$
            \item $C[s_1,\cdots,s_n]$ denotes $C$ with each $\Box_i$ replaced by corresponding $s_i$
        \end{itemize}
    \end{definition}
\end{frame}

\begin{frame}
    \frametitle{}

    \begin{tabular}{ll}
        $\m{sum}(x) \to \m{sum2}(\m{even}(x),x)$ & \\
        $\m{sum2}(\m{true}, x) \to \m{sum3}(x)$ & $\m{sum2}(\m{false}, x)$ \to $\m{sum3}(\m{p}(x))$\\
        $\m{sum3}(\m{s}(\m{s}(x))) \to \m{add}(\m{s}(\m{s}(x)), \m{sum3}(x))$ & $\m{sum3}(\m{0}) \to \m{0}$\\\\\\
        $\m{even}(\m{s}(\m{s}(x))) \to \m{even}(x)$ & $\m{even}(\m{s}(x)) \to \m{false}$\\
        $\m{even}(\m{0}) \to \m{true}$ & \\
        $\m{p}(\m{s}(x)) \to x$ & $\m{p}(\m{0}) \to \m{0}$\\
        $\m{add}(x,\m{s}(y)) \to \m{s}(\m{add}(x,y))$ & $\m{add}(x,\m{0}) \to x$
        \end{tabular}

\end{frame}

\begin{frame}
    \frametitle{$\Sterms$ and $\Stheory$}
    from now, consider two signatures $\Sterms$ and $\Stheory$
    \begin{example}
        \begin{itemize}
            \item $\Sterms$\\
                \begin{tabular}{lll}
                    $\m{sum} : [\m{int}] \Rightarrow \m{int}$ & $\m{sum2} : [\m{bool} \times \m{int}] \Rightarrow \m{int}$ & $\m{sum3} : [\m{int}] \Rightarrow \m{int}$
                \end{tabular}
            \item $\Stheory$\\
                \begin{tabular}{lll}
                    $\m{n} : \m{int}$ (for all $n \in \mathbb{Z}$) & $\m{true} : \m{bool}$ & $\m{false} : \m{bool}$\\
                    $+ : [\m{int} \times \m{int}] \Rightarrow \m{int}$ & $- : [\m{int} \times \m{int}] \Rightarrow \m{int}$ &\\
                    $\geq : [\m{int} \times \m{int}] \Rightarrow \m{bool}$ & $\m{even}: [\m{int} \times \m{int}] \Rightarrow \m{bool}$ & $\neg : [\m{bool}] \Rightarrow \m{bool}$
                \end{tabular}
        \end{itemize}
    \end{example}
\end{frame}


\begin{frame}
    \frametitle{Logical Term}

    \begin{definition}[proper and logical term]
        for term $s \in \Terms(\Sterms \cup \Stheory, \VV)$
        \begin{itemize}
            \item $s$ is \textbf{proper term} if $s \in \Terms(\Sterms, \VV)$
            \item $s$ is \textbf{logical term} if $s \in \Terms(\Stheory, \VV)$
            \item logical term is \textbf{value} if it is form of $s : \kappa$
        \end{itemize}
    \end{definition}

    \begin{example}
        \begin{itemize}
            \item $\m{0},\m{true}$ logical term and value
            \item $\m{x} + 1$ logical term
            \item $\m{sum2}(\m{true},x)$ proper term
            \item $\m{sum}{(\m{1} + \m{1})}$ mixed term
        \end{itemize}
    \end{example}
\end{frame}

\begin{frame}
    \frametitle{Interpretation of $\Stheory$}
    \begin{definition}[interpretation mapping$\JJ$]
        assume mapping $\II_\iota$ from occurring sort $\iota$ in $\Stheory$ to corresponding set
        \begin{itemize}
        \item \textbf{interpretation mapping} $\JJ$ is mapping from each
            $f : [\iota_1 \times \cdots \times \iota_n] \in \Stheory$ to function $\JJ_f \in \II_{\iota_1} \Rightarrow \cdots \II{\iota_n} \Rightarrow \II{\kappa}$
        \item interpretation $\JJ$ can be extended to interpretation of ground logical term
            $\interpret{f(s_1,\cdot,s_n)} = \JJ_f(\interpret{s_1},\cdots,\interpret{s_n})$
        \end{itemize}
    \end{definition}

    \begin{example}
        choose $\II(\m{int}) = \mathbb{Z}$ and $\II(\m{bool}) = \mathbb{B} = \{\top, \perp \}$
        \begin{itemize}
            \item $\JJ(\m{+} : [\m{int} \times \m{int}] \Rightarrow \m{int}) = \lambda x y.x + y$
            \item $\JJ(\m{even} : [\m{int}] \Rightarrow \m{bool}) = \lambda x.\text{''$x$ is even number"}$
        \end{itemize}
    \end{example}
\end{frame}

\begin{frame}
        \begin{align*}
        \m{sum}(x) &\to \m{sum2}(\m{even}(x),x) \\
        \m{sum2}(\m{true}, x) &\to \m{sum3}(\neg(0 \geq x), x) \\
        \m{sum2}(\m{false}, x) &\to \m{sum3}(\neg(0 \geq (x-\m{1})), x - \m{1}))\\
        \m{sum3}(\m{true}, x) &\to x + \m{sum3}(\neg(0 \geq x-2), x-2) \\
        \m{sum3}(\m{false},x) &\to \m{0}\\
        \end{align*}
        \vspace{-10mm}
        \begin{align*}
            \m{sum}(x) &\to \m{sum2}(x) &\\
            \m{sum2}(x) &\to \m{sum3}(x) & \text{if} & \;\; \m{even}(x)\\
            \m{sum2}(x) &\to \m{sum3}(x-\m{1}) & \text{if} & \;\; \neg(\m{even}(x))\\
            \m{sum3}(x) &\to x + {sum3}(x-2) & \text{if} & \;\; \neg(0 \geq x)  \\
            \m{sum3}(x) &\to \m{0} & \text{if} & \;\;0 \geq x \\
        \end{align*}
\end{frame}

\begin{frame}
    \frametitle{Logical Constraint}
    \begin{definition}[logical constraint]
        \begin{itemize}
        \item \textbf{logical constraint} is logical term of sort $\m{bool}$ with $\mathcal{I}_{\m{bool}} = \mathbb{B}$
        \item logical constraint $\varphi$ is \textbf{valid} if $\llbracket \varphi\gamma \rrbracket_\JJ = \top$ for all substitution $\gamma$
        \end{itemize}
    \end{definition}
    \begin{example}
        \begin{itemize}
            \item $\m{even}(\m{x} + \m{1})$ logical constraint
            \item $\m{5} + {3}$ not logical constraint
            \item $\m{true}$ logical constraint
        \end{itemize}
    \end{example}

\end{frame}

\begin{frame}
    \frametitle{Rewrite Rule}
    \begin{definition}[rewrite rule]
        let $l$ and $r$ be term and $\varphi$ be logical constraint
        \begin{itemize}
            \item \textbf{rewrite rule} is triple $l \to r \; [\varphi]$
            \item $l$ and $r$ must be same sort
            \item denote $l \to r$ if $l \to r \; [\m{true}]$
            \item $\LVar(l \to r \; [\varphi])$ is defined as $\Var(\varphi) \cup (\Var(r) \setminus \Var(l))$
        \end{itemize}
    \end{definition}
    \begin{itemize}
        \item $\m{sum}(x) \to \m{sum2}(x)$ rule
        \item $\m{sum2}(x) \to \m{sum3}(x-1) \; [\neg(\m{even}(x))]$ rule
        \item $\m{sum}(x) \to \m{true} \; [\m{even}(x)]$ not rule
    \end{itemize}
\end{frame}

\begin{frame}
    \frametitle{Logically Constrained Term Rewriting System}
    \begin{definition}[LCTRS]
        \textbf{logical constrained term rewriting system} is set of rewrite rules
    \end{definition}

    \begin{example}
        \begin{tabular}{ll}
            $\m{sum}(x) \to \m{sum2}(x)$ & \\
            $\m{sum2}(x) \to \m{sum3}(x) \; [\m{even}(x)]$ & $\m{sum2}(x) \to \m{sum3}(x-1) \; [\neg(\m{even}(x))]$ \\
            $\m{sum3}(x) \to x + \m{sum3}(x) \; [0 \leq x]$ & $\m{sum3}(x) \to \m{0} \; [\neg(0 \leq x)]$
        \end{tabular}
    \end{example}
\end{frame}

\begin{frame}
    \frametitle{Rewrite Relation $\to_\RR$}
    \begin{definition}
        substitution $\gamma$ \textbf{respects} rule $l \to r \; [\varphi]$ if
        \begin{itemize}
            \item $Dom(\gamma) = \Var(l) \cup \Var(r) \cup \Var(\varphi)$
            \item $\gamma(x)$ is value for all $x \in \LVar(l \to r \; [\varphi])$
            \item $\varphi\gamma$ is valid
        \end{itemize}
    \end{definition}
    \begin{definition}[rewrite relation $\to_\RR$]
        \textbf{rewrite relation} $\to_\RR$ is union of $\torule$ and $\tocalc$ where
        \begin{itemize}
            \item $C[l\gamma] \torule C[r\gamma]$ if $l \to r \; [\varphi] \in \RR$ and $\gamma$ respects $l \to r \; [\varphi]$
            \item $C[f(s_1,\cdots,s_n)] \tocalc C[v]$ if $f \in \Stheory\setminus\Sterms$, all $s_i$ values and $v$ is value of $f(s_1,\cdots,s_n)$
        \end{itemize}
    \end{definition}
\end{frame}

\begin{frame}
    \begin{example}
        \begin{tabular}{ll}
            $\m{sum}(x) \to \m{sum2}(x)$ & \\
            $\m{sum2}(x) \to \m{sum3}(x) \; [\m{even}(x)]$ & $\m{sum2}(x) \to \m{sum3}(x-1) \; [\neg(\m{even}(x))]$ \\
            $\m{sum3}(x) \to x + \m{sum3}(x) \; [\neg(0 \geq x)]$ & $\m{sum3}(x) \to \m{0} \; [0 \geq x]$
        \end{tabular}
        \begin{align*}
            & \m{sum}(\m{5}) \torule \m{sum2}(\m{5}) \torule \m{sum3}(\m{5} - \m{1}) \tocalc \m{sum3}(\m{4})\\
            & \torule \m{4} + \m{sum3}(\m{4} -\m{2}) \tocalc \m{4} + \m{sum3}(\m{2}) \torule \m{4} + (\m{2} + \m{sum3}(\m{2}-\m{2}))\\
            & \tocalc \m{4} + (\m{2} + \m{sum3}(0)) \torule \m{4} + (\m{2} + \m{0}) \tocalc \m{4} + \m{2} \\
            & \tocalc \m{6}
        \end{align*}
    \end{example}

\end{frame}

\begin{comment}


\begin{frame}
    \begin{example}
        \begin{tabular}{ll}
            $\m{sum} : [\m{int}] \Rightarrow \m{int}$ & $\m{sum2} [\m{bool} \times \m{int}] \Rightarrow \m{int}$ \\
            $\m{0} : \m{int}$ & $\m{s} : [\m{int}] \Rightarrow \m{int}$\\
            $\m{true} : \m{bool}$ & $\m{false} :\m{bool}$
        \end{tabular}

        \begin{itemize}
            \item $\m{sum}(\m{0})$
            \item $\m{sum2}(\m{true}, \m{s}(\m{0}))$
            \item $\m{s}(\m{false})$
            \item $\m{sum2}(\m{s}(\m{0}) ,\m{ture})$
        \end{itemize}

    \end{example}
\end{frame}

\begin{frame}
    \begin{example}
    \begin{tabular}{ll}
        $\m{sum}(x) \to \m{sum2}(x) \; [x \bmod 2 = 0]$ & $\m{sum}(x) \to \m{sum2}(x - 1) \; [x \bmod 2 = 1]$\\
        $\m{sum2}(x) \to x + \m{sum2}(x-2) \; [\neg(0 \geq x)]$ & $\m{sum2}(x) \to 0 \; [0 \geq x]$
    \end{tabular}
    \end{example}
    \begin{example}
    \begin{tabular}{l}
        $\m{sum}(x) \to x + \m{sum}(x - \m{2}) \; [\neg(\m{0} \geq x) \land \m{even}(x)]$ \\
        $\m{sum}(x) \to \m{sum}(x - \m{1}) \; [\neg(\m{0} \geq x) \land \neg(\m{even}(x))]$\\
        $\m{sum}(x) \to \m{0} \; [\m{0} \geq x]$
    \end{tabular}

        where $\mathcal{J}_{\m{even}} = \lambda x.\text{''$x$ is even number"}$
    \end{example}

\end{frame}

\begin{frame}
    \begin{example}
        \begin{align*}
            \m{sum}(\m{4}) & \torule \m{sum2}(\m{4}) \torule \m{4} + \m{sum2}(\m{4}-\m{2}) \tocalc \m{4} + \m{sum2}(\m{2}) \\
             & \torule \m{4} + (\m{2} + \m{sum2}(\m{2} - \m{0})) \tocalc \m{4} + (\m{2} + \m{sum2}(\m{0})) \\
             & \torule \m{4} + (\m{2} + \m{0}) \tocalc \m{4} + \m{2} \tocalc \m{6}
        \end{align*}
    \end{example}
\end{frame}

\end{comment}

\end{document}
