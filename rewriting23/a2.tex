% 4up version
%\documentclass[handout, 12pt, aspectratio=169]{beamer}

\documentclass[12pt,aspectratio=169]{beamer}

\usefonttheme{professionalfonts}

\mode<handout>
{
\usepackage{pgfpages}
\pgfpagesuselayout{4 on 1}[a4paper,landscape]
}

\usepackage[english]{babel}
\usepackage{luatexja}
\usepackage{luatexja-fontspec}


% for display code
%\usepackage{minted}
%\newfontfamily\hasklig{hasklig}[NFSSFamily=haskligFamily]
%\setminted[haskell]{fontfamily=haskligFamily,
%mathescape,
%       numbersep=5pt
%}

% for proof tree
%\usepackage{bcprules}

%\usepackage[T1]{fontenc}

%for lualatex
\usepackage{fontspec}
\setsansfont{CMU Sans Serif}%{Arial}
\setmainfont{CMU Serif}%{Times New Roman}
\setmonofont{CMU Typewriter Text}%{Consolas}

\usepackage{lmodern,amsmath,amssymb,proof}
\usepackage{tikz}
\usetikzlibrary{arrows,positioning}



% color scheme
\definecolor{green}{rgb}{0.0,0.5,0.0}
\definecolor{blue}{rgb}{0.0,0.0,0.7}
\definecolor{red}{rgb}{0.8,0.0,0.0}
\definecolor{lightred}{rgb}{1.0,0.97,0.97}
\definecolor{lightblue}{rgb}{0.95,0.95,1.0}
\definecolor{darkblue}{rgb}{0.3,0.3,0.5}
\definecolor{darkred}{rgb}{0.6,0.0,0.0}
\definecolor{darkorange}{rgb}{0.6,0.2,0.1}
\definecolor{darkgreen}{rgb}{0,0.3,0}
%
\newcommand{\IMAGE}[2]{\pgfdeclareimage[#1]{#2}{#2}\pgfuseimage{#2}}

\setbeamerfont{title}{series=\bfseries,size=\Large}
\setbeamercolor{title}{fg=red}
\setbeamerfont{subtitle}{series=\bfseries,size=\LARGE}
\setbeamercolor{subtitle}{fg=red}
\setbeamerfont{author}{series=\bfseries,size=\large}
\setbeamercolor{author}{fg=blue}
\setbeamerfont{institute}{series=\bfseries,size=\large}
\setbeamercolor{institute}{fg=green}

\setbeamertemplate{blocks}[rounded]
\setbeamerfont{block title}{series=\bfseries,family=\sffamily,size=\small\strut}
\setbeamercolor{block title}{bg=darkblue,fg=white}
\setbeamercolor{block body}{bg=blue!4!white}
\setbeamercolor{block title example}{bg=white,fg=darkgreen}
\setbeamercolor{block body example}{bg=white}
\setbeamercolor{block title alerted}{bg=darkred,fg=white}
\setbeamercolor{block body alerted}{bg=red!4!white}
\setbeamercolor{structure}{fg=darkred}
% no fading effect
\makeatletter
\pgfdeclareverticalshading[lower.bg,upper.bg]{bmb@transition}{200cm}{%
  color(0pt)=(lower.bg); color(4pt)=(lower.bg); color(4pt)=(upper.bg)}
\makeatother

% frametitle
\useframetitletemplate{
\begin{centering}
\centerline{\large\bfseries\color{darkblue}\insertframetitle}
\end{centering}
}

% footer  (title  page/pages)
\setbeamertemplate{navigation symbols}{}
\useheadtemplate{\vbox{\vskip8pt}}
\usefoottemplate{\vbox{\vskip2pt\inserttitle\hfil\insertframenumber/\inserttotalframenumber\vskip5pt}}

% enumerate/itemize environment
\newcommand*\tikzboxed[1]{\tikz[baseline=(c.base)]{%
\node[thick,shape=rectangle,draw,inner sep=2pt] (c) {#1};}}
\setbeamertemplate{items}[square]
\setbeamertemplate{enumerate item}{\tikzboxed{\footnotesize\insertenumlabel}}
\setlength{\itemsep}{1ex}

\newcommand{\m}[1]{\mathsf{#1}}
\newcommand{\mi}[1]{\mathit{#1}}
\newcommand{\md}[1]{\mathtt{\textcolor{blue}{#1}}}
\newcommand{\seq}[2][n]{{#2_1},\dots,{#2_{#1}}}
%
\newcommand{\FF}{\mathcal{F}}
\newcommand{\RR}{\mathcal{R}}
\newcommand{\VV}{\mathcal{V}}
\newcommand{\TT}{\mathcal{T}}
\newcommand{\Var}{\mathcal{V}\m{ar}}
%
\newcommand{\app}{\circ}

\title{ Term Rewriting / A2.Well-founded Induction }
\author{Wataru Yachi}
\institute{JAIST}
\date{Oct 26, 2023}

\begin{document}

\maketitle



\begin{frame}
    \frametitle{Relation Admits Induction}
    \begin{definition}
        relation $R$ on set $A$ \alert{admits induction} if
        property $P$ holds for all elements in $A$ whenever
        element $a \in A$ has property $P$ if for all $b \in A$
        with $a \; R \; b$, $P(b)$ holds
    \end{definition}

    \begin{example}
        successor relation $>_1$ on $\mathbb{N}$ admits induction
    \end{example}
\end{frame}

\begin{frame}
    \frametitle{Minimal Element}

    \begin{definition}
        let $R$ be relation on $A$ and let $B$ be subset of $A$,
        element $a \in B$ is called \alert{minimal} element of $B$
        if there is no $b \in B$ such that $a \; R \; b$
    \end{definition}

    \begin{example}
    \end{example}
\end{frame}

\begin{frame}
    \frametitle{Well-founded Induction}

    \begin{theorem}
        let $R$ be relation on $A$, following statements are equivalent
        \begin{enumerate}
            \item $R$ admits induction
            \item there are no infinite descending sequence $a_1 \;R \; a_2 \; R \; a_3 \; R \; \cdots$
            \item every non-empty subset $B$ of $A$ contains a minimal element
        \end{enumerate}
    \end{theorem}

    \begin{definition}
        relation is called \alert{well-founded} if it
        satisfied one of above assertions
    \end{definition}
\end{frame}

\begin{frame}
    \frametitle{}
    \begin{definition}
        \begin{itemize}
            \item well-founded strict order is called \alert{well-founded order}
            \item partial order is \alert{well-founded} if its strict part is well-founded
            \item \alert{well-order} is total well-founded order
        \end{itemize}
    \end{definition}
\end{frame}

\begin{frame}
    \frametitle{}
    \begin{lemma}
        every strict order on finite set is well-founded
    \end{lemma}

    \begin{proof}
        We show the claim by contradiction.
        Let $>$ be a strict order on a finite set $A$.
        Suppose an infinite descending chain $a_1 > a_2 > a_3 > \cdots$
        of elements of $A$.
        Because $A$ is finite, some element $a$ appears at least two times in the chain. Thus $a > a$ since the transitivity of $>$.
        By the irreflexivity of $>$, there is no $a \in A$ such that $a > a$. So we have contradiction.
    \end{proof}

    \begin{theorem}
        every well-founded order can be extended to well-order on same set
    \end{theorem}
\end{frame}

\begin{frame}
    \frametitle{}
    \begin{theorem}
        lexicographic product of well-founded orders is well-founded
    \end{theorem}

    \begin{proof}
        Let $>_1$ and $>_2$ be a well-founded order on $A$ and $B$ respectively. Fix $> \: = (>_1,>_2)_{\mathsf{lex}}$ on $A \times B$.
        Suppose an infinite descending chain $(a_1, b_1) > (a_2, b_2) > (a_3, b_3) > \cdots$ of elements of $A \times B$.
    \end{proof}
\end{frame}

\end{document}
