% 4up version
%\documentclass[handout, 12pt, aspectratio=169]{beamer}

\documentclass[12pt,aspectratio=169]{beamer}

\usefonttheme{professionalfonts}

\mode<handout>
{
\usepackage{pgfpages}
\pgfpagesuselayout{4 on 1}[a4paper,landscape]
}

\usepackage[english]{babel}
\usepackage{luatexja}
\usepackage{luatexja-fontspec}


% for display code
%\usepackage{minted}
%\newfontfamily\hasklig{hasklig}[NFSSFamily=haskligFamily]
%\setminted[haskell]{fontfamily=haskligFamily,
%mathescape,
%       numbersep=5pt
%}

% for proof tree
%\usepackage{bcprules}

%\usepackage[T1]{fontenc}

%for lualatex
\usepackage{fontspec}
\setsansfont{CMU Sans Serif}%{Arial}
\setmainfont{CMU Serif}%{Times New Roman}
\setmonofont{CMU Typewriter Text}%{Consolas}

\usepackage{lmodern,amsmath,amssymb,proof}
\usepackage{tikz}
\usetikzlibrary{arrows,positioning}



% color scheme
\definecolor{green}{rgb}{0.0,0.5,0.0}
\definecolor{blue}{rgb}{0.0,0.0,0.7}
\definecolor{red}{rgb}{0.8,0.0,0.0}
\definecolor{lightred}{rgb}{1.0,0.97,0.97}
\definecolor{lightblue}{rgb}{0.95,0.95,1.0}
\definecolor{darkblue}{rgb}{0.3,0.3,0.5}
\definecolor{darkred}{rgb}{0.6,0.0,0.0}
\definecolor{darkorange}{rgb}{0.6,0.2,0.1}
\definecolor{darkgreen}{rgb}{0,0.3,0}
%
\newcommand{\IMAGE}[2]{\pgfdeclareimage[#1]{#2}{#2}\pgfuseimage{#2}}

\setbeamerfont{title}{series=\bfseries,size=\Large}
\setbeamercolor{title}{fg=red}
\setbeamerfont{subtitle}{series=\bfseries,size=\LARGE}
\setbeamercolor{subtitle}{fg=red}
\setbeamerfont{author}{series=\bfseries,size=\large}
\setbeamercolor{author}{fg=blue}
\setbeamerfont{institute}{series=\bfseries,size=\large}
\setbeamercolor{institute}{fg=green}

\setbeamertemplate{blocks}[rounded]
\setbeamerfont{block title}{series=\bfseries,family=\sffamily,size=\small\strut}
\setbeamercolor{block title}{bg=darkblue,fg=white}
\setbeamercolor{block body}{bg=blue!4!white}
\setbeamercolor{block title example}{bg=white,fg=darkgreen}
\setbeamercolor{block body example}{bg=white}
\setbeamercolor{block title alerted}{bg=darkred,fg=white}
\setbeamercolor{block body alerted}{bg=red!4!white}
\setbeamercolor{structure}{fg=darkred}
% no fading effect
\makeatletter
\pgfdeclareverticalshading[lower.bg,upper.bg]{bmb@transition}{200cm}{%
  color(0pt)=(lower.bg); color(4pt)=(lower.bg); color(4pt)=(upper.bg)}
\makeatother

% frametitle
\useframetitletemplate{
\begin{centering}
\centerline{\large\bfseries\color{darkblue}\insertframetitle}
\end{centering}
}

% footer  (title  page/pages)
\setbeamertemplate{navigation symbols}{}
\useheadtemplate{\vbox{\vskip8pt}}
\usefoottemplate{\vbox{\vskip2pt\inserttitle\hfil\insertframenumber/\inserttotalframenumber\vskip5pt}}

% enumerate/itemize environment
\newcommand*\tikzboxed[1]{\tikz[baseline=(c.base)]{%
\node[thick,shape=rectangle,draw,inner sep=2pt] (c) {#1};}}
\setbeamertemplate{items}[square]
\setbeamertemplate{enumerate item}{\tikzboxed{\footnotesize\insertenumlabel}}
\setlength{\itemsep}{1ex}

\newcommand{\m}[1]{\mathsf{#1}}
\newcommand{\mi}[1]{\mathit{#1}}
\newcommand{\md}[1]{\mathtt{\textcolor{blue}{#1}}}
\newcommand{\seq}[2][n]{{#2_1},\dots,{#2_{#1}}}
%
\newcommand{\FF}{\mathcal{F}}
\newcommand{\RR}{\mathcal{R}}
\newcommand{\VV}{\mathcal{V}}
\newcommand{\TT}{\mathcal{T}}
\newcommand{\Var}{\mathcal{V}\m{ar}}
%
\newcommand{\app}{\circ}
\newcommand{\CA}{\mathcal{A}}

\title{Term Rewriting}
\subtitle{1.1 Definitions}
\author{Wataru Yachi}
\institute{JAIST}
\date{Nov 9, 2023}

\begin{document}

\newcommand{\fig}{
        \begin{tikzpicture}
            \node (a) at (0,0) {$\mathsf{a}$}; \node (b) at (1,0) {$\mathsf{b}$}; \node (c) at (2,0) {$\mathsf{c}$}; \node (g) at (3,0) {$\mathsf{g}$};
            \node (d) at (0,-1) {$\mathsf{d}$}; \node (e) at (1,-1) {$\mathsf{e}$};
            \node (f) at (0,-2) {$\mathsf{f}$};

            \path[->, >=stealth]
                (a) edge (b)
                (b) edge (c)
                %(c) edge (g)
                (d) edge (a)
                (d) edge (e)
                (b) edge (e)
                %(c) edge (e)
                (d) edge (f)
                (e) edge (f)
                (g) edge[loop right] (g);
        \end{tikzpicture}

}

\maketitle

\begin{frame}
    \frametitle{Abstract Rewrite System}
    \begin{definition}
        \pause
        \begin{itemize}[<+->]
            \item \alert{abstract rewrite system} $\CA$ is pair $(A, \rightarrow)$ of set $A$ and relation $\rightarrow$ on $A$
            \item \alert{$a \rightarrow b$} denotes $(a,b) \in \; \rightarrow$
        \end{itemize}
    \end{definition}
    \pause
    \begin{example}
        \begin{columns}
            \begin{column}{0.48\textwidth}
                let $\mathcal{A} = (A,\rightarrow)$ with
                \begin{align*}
                    &&A &= \{\m{a}, \m{b}, \m{c}, \m{d}, \m{e}, \m{f}\}\\
                    &&\rightarrow\; &=
                    \left\{ \begin{array}{llll}
                        (\m{a},\m{b}), & (\m{b},\m{c}),& (\m{b}, \m{e}), & (\m{d}, \m{a}), \\
                        (\m{d}, \m{e}), & (\m{d},\m{f}), & (\m{e}, \m{f}), & (\m{g}, \m{g})
                    \end{array}\right\}
                \end{align*}
        \end{column}\pause
        \begin{column}{0.35\textwidth}
            \centering
            \fig
        \end{column}
        \end{columns}
    \end{example}
\end{frame}

\begin{frame}
    \frametitle{Composition of $\rightarrow$}
    \begin{definition}\pause
        \begin{itemize}
            %\item \alert{$a \rightarrow \cdot \rightarrow b$} if $a \rightarrow c$ and $c\rightarrow b$ for some c
            \pause
            \item \alert{$a \rightarrow^{0} b$} if $a = b$
            \pause
            \item \alert{$a \rightarrow^{n+1}$} if $a \rightarrow^{n} \cdot \rightarrow b$
        \end{itemize}
    \end{definition}
    \pause
    \begin{example}
        \begin{columns}
        \begin{column}{0.35\textwidth}
            \centering
            \fig
        \end{column}
        \begin{column}{0.5\textwidth}
            \pause
            \begin{itemize}
                \item $\m{f} \rightarrow^{0} \m{f}$ \pause because $f=f$
                \item $\m{a} \rightarrow^{2} \m{c}$ \pause because $\m{a} \rightarrow \m{b} \rightarrow \m{c}$
                \item $\m{g} \rightarrow^{1000} \m{g}$ \pause  because $\m{g}\rightarrow \cdots \rightarrow \m{g}$
            \end{itemize}
        \end{column}
        \end{columns}
    \end{example}
\end{frame}


\begin{frame}
    \frametitle{Closures of $\rightarrow$}
    \begin{definition}
        \pause
        \begin{itemize}
            \item \alert{$a \rightarrow^{=} b$} if $a\rightarrow b$ or $a = b$ \hfill \alert{reflexive closure of $\rightarrow$}
            \pause \item \alert{$a \rightarrow^{+} b$} if $a\rightarrow^{n} b$ for some $n \geq 1$ \hfill \alert{transitive closure of $\rightarrow$}
            \pause \item \alert{$a \rightarrow^{*} b$} if $a\rightarrow^{n} b$ for some $n \geq 0$ \hfill \alert{reflexive and transitive closure $\rightarrow$}
        \end{itemize}
    \end{definition}
\pause
    \begin{example}
    \begin{columns}
        \begin{column}{0.35\textwidth}
            \centering
            \fig
        \end{column}
    \pause
        \begin{column}{0.5\textwidth}
            \begin{itemize}
                \item $\m{a} \rightarrow^{=} \m{b}$
                \pause \item $\m{a} \rightarrow^{*} \m{f}$ and also $\m{a} \rightarrow^{+} \m{f}$
                \pause \item $\m{a} \rightarrow^{*} \m{a}$ but $\m{a} \rightarrow^{+} \m{a}$ does not hold
                \pause \item $\m{f} \rightarrow^{*} \m{g}$ does not hold
            \end{itemize}
        \end{column}
        \end{columns}
    \end{example}
\end{frame}

\begin{frame}
    \frametitle{Conversion}
    \begin{definition}
        \begin{itemize}
            \pause
            \item \alert{$a \leftarrow b$} if $b \rightarrow a$ \hfill \alert{inverse of $\rightarrow$}
            \pause
            \item \alert{$a \leftrightarrow b$} if $a \rightarrow b$ or $a \leftarrow b$ \hfill \alert{symmetric closure of $\rightarrow$}
            \pause
            \item \alert{$a \leftrightarrow^{*} b$} if $a \leftrightarrow^{n} b$ for some $n \geq 0$ \hfill \alert{conversion}
        \end{itemize}
    \end{definition}
    \pause

    \begin{example}
    \begin{columns}
        \begin{column}{0.35\textwidth}
            \centering
            \fig
        \end{column}
        \begin{column}{0.5\textwidth}
            \begin{itemize}
                \pause
                \item $\m{b} \leftrightarrow^{*} \m{d}$ \pause because $\m{d} \rightarrow \m{e} \leftarrow \m{b}$
                \pause
                \item $\m{a} \leftrightarrow^{*} \m{a}$
                \pause
                \item $\m{f} \leftrightarrow^{*} \m{g}$ does not hold
            \end{itemize}
        \end{column}
        \end{columns}
    \end{example}

\end{frame}

\begin{frame}
    \frametitle{Joinability and Meetability}
    \begin{definition}
        \begin{itemize}
                \pause
            \item \alert{$a \downarrow b$} if $a \rightarrow^* c \;\reflectbox{$\rightarrow^*$}\; b$ for some $c$ \hfill \alert{joinability}
                \pause
            \item \alert{$a \uparrow b$} if $a \;\reflectbox{$\rightarrow^*$}\; c \rightarrow^* b$ for some $c$ \hfill \alert{meetability}
        \end{itemize}
    \end{definition}
\pause
    \begin{example}
    \begin{columns}
        \begin{column}{0.35\textwidth}
            \centering
            \fig
        \end{column}
        \begin{column}{0.5\textwidth}
            \begin{itemize}
                \pause
                \item $\m{a} \downarrow \m{d}$ \pause because $\m{a} \rightarrow \m{b} \rightarrow \m{e} \rightarrow \m{f} \leftarrow \m{d}$
                \pause
                \item $\m{f} \uparrow \m{c}$ \pause because $\m{f} \leftarrow \m{e} \leftarrow \m{b} \leftarrow \m{a} \rightarrow \m{b} \rightarrow \m{c}$
                \pause
                \item $\m{a} \downarrow \m{g}$ \pause does not hold
                \pause
                \item $\m{c} \downarrow \m{d}$ \pause but $\m{c} \uparrow \m{d}$ does not hold
            \end{itemize}
        \end{column}
        \end{columns}
    \end{example}
\end{frame}


\begin{frame}
    \frametitle{Normal Form}
    \begin{definition}
        \begin{itemize}
            \pause
            \item $a$ is \alert{normal form} if there is no $b$ such that $a \rightarrow b$
            \pause 
            \item \alert{$\mathsf{NF}(\mathcal{A})$} denotes set of all normal forms of $\mathcal{A}$
            \pause
            \item \alert{$a \rightarrow^{!} b$} if $a \rightarrow^* b$ and $b \in \mathsf{NF}(\mathcal{A})$
        \end{itemize}
    \end{definition}
\pause
    \begin{example}
    \begin{columns}
        \begin{column}{0.35\textwidth}
            \centering
            \fig
        \end{column}
        \begin{column}{0.5\textwidth}
            \begin{itemize}
                \pause
                \item $\mathsf{NF}(\mathcal{A}) = \{\m{c}, \m{f}\}$
                \pause
                \item $\m{a} \rightarrow^! \m{f}$
                \pause
                \item $\m{a} \rightarrow^! \m{c}$
                \pause
                \item $\m{g}$ has no normal form
            \end{itemize}
        \end{column}
        \end{columns}
    \end{example}

\end{frame}


\end{document}
