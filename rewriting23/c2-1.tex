% 4up version
%\documentclass[handout, 12pt, aspectratio=169]{beamer}

\documentclass[12pt,aspectratio=169]{beamer}

\usefonttheme{professionalfonts}

\mode<handout>
{
\usepackage{pgfpages}
\pgfpagesuselayout{4 on 1}[a4paper,landscape]
}

\usepackage[english]{babel}
\usepackage{luatexja}
\usepackage{luatexja-fontspec}


% for display code
%\usepackage{minted}
%\newfontfamily\hasklig{hasklig}[NFSSFamily=haskligFamily]
%\setminted[haskell]{fontfamily=haskligFamily,
%mathescape,
%       numbersep=5pt
%}

% for proof tree
%\usepackage{bcprules}

%\usepackage[T1]{fontenc}

%for lualatex
\usepackage{fontspec}
\setsansfont{CMU Sans Serif}%{Arial}
\setmainfont{CMU Serif}%{Times New Roman}
\setmonofont{CMU Typewriter Text}%{Consolas}

\usepackage{lmodern,amsmath,amssymb,proof}
\usepackage{latexsym}
\usepackage{tikz}
\usetikzlibrary{arrows,positioning}



% color scheme
\definecolor{green}{rgb}{0.0,0.5,0.0}
\definecolor{blue}{rgb}{0.0,0.0,0.7}
\definecolor{red}{rgb}{0.8,0.0,0.0}
\definecolor{lightred}{rgb}{1.0,0.97,0.97}
\definecolor{lightblue}{rgb}{0.95,0.95,1.0}
\definecolor{darkblue}{rgb}{0.3,0.3,0.5}
\definecolor{darkred}{rgb}{0.6,0.0,0.0}
\definecolor{darkorange}{rgb}{0.6,0.2,0.1}
\definecolor{darkgreen}{rgb}{0,0.3,0}
%
\newcommand{\IMAGE}[2]{\pgfdeclareimage[#1]{#2}{#2}\pgfuseimage{#2}}

\setbeamerfont{title}{series=\bfseries,size=\Large}
\setbeamercolor{title}{fg=red}
\setbeamerfont{subtitle}{series=\bfseries,size=\LARGE}
\setbeamercolor{subtitle}{fg=red}
\setbeamerfont{author}{series=\bfseries,size=\large}
\setbeamercolor{author}{fg=blue}
\setbeamerfont{institute}{series=\bfseries,size=\large}
\setbeamercolor{institute}{fg=green}

\setbeamertemplate{blocks}[rounded]
\setbeamerfont{block title}{series=\bfseries,family=\sffamily,size=\small\strut}
\setbeamercolor{block title}{bg=darkblue,fg=white}
\setbeamercolor{block body}{bg=blue!4!white}
\setbeamercolor{block title example}{bg=white,fg=darkgreen}
\setbeamercolor{block body example}{bg=white}
\setbeamercolor{block title alerted}{bg=darkred,fg=white}
\setbeamercolor{block body alerted}{bg=red!4!white}
\setbeamercolor{structure}{fg=darkred}
% no fading effect
\makeatletter
\pgfdeclareverticalshading[lower.bg,upper.bg]{bmb@transition}{200cm}{%
  color(0pt)=(lower.bg); color(4pt)=(lower.bg); color(4pt)=(upper.bg)}
\makeatother

% frametitle
\useframetitletemplate{
\begin{centering}
\centerline{\large\bfseries\color{darkblue}\insertframetitle}
\end{centering}
}

% footer  (title  page/pages)
\setbeamertemplate{navigation symbols}{}
\useheadtemplate{\vbox{\vskip8pt}}
\usefoottemplate{\vbox{\vskip2pt\inserttitle\hfil\insertframenumber/\inserttotalframenumber\vskip5pt}}

% enumerate/itemize environment
\newcommand*\tikzboxed[1]{\tikz[baseline=(c.base)]{%
\node[thick,shape=rectangle,draw,inner sep=2pt] (c) {#1};}}
\setbeamertemplate{items}[square]
\setbeamertemplate{enumerate item}{\tikzboxed{\footnotesize\insertenumlabel}}
\setlength{\itemsep}{1ex}

\newcommand{\m}[1]{\mathsf{#1}}
\newcommand{\mi}[1]{\mathit{#1}}
\newcommand{\md}[1]{\mathtt{\textcolor{blue}{#1}}}
\newcommand{\seq}[2][n]{{#2_1},\dots,{#2_{#1}}}
%
\newcommand{\FF}{\mathcal{F}}
\newcommand{\RR}{\mathcal{R}}
\newcommand{\VV}{\mathcal{V}}
\newcommand{\TT}{\mathcal{T}}
\newcommand{\Var}{\mathcal{V}\m{ar}}
%
\newcommand{\app}{\circ}

\newlength{\mytotalwidth}
\mytotalwidth=\dimexpr\linewidth-5mm
\newlength{\mycolumnwidth}
\mycolumnwidth=\dimexpr\mytotalwidth-5mm

\title{ Terms }
\author{Wataru Yachi}
\institute{JAIST}
\date{January 22, 2024}

\begin{document}

\maketitle

\begin{frame}
    \frametitle{Signatures}
    \begin{definition}
        \begin{itemize}
            \item \alert{signature} $\FF$ is set of function symbols
            \item \alert{$f^{(n)}$} denotes $f$ will take $n$ arguments (n is called \alert{arity})
            \item $f$ is \alert{constant} if arity of $f$ is 0
        \end{itemize}
    \end{definition}
    
    \pause
    \begin{example}
        \begin{columns}[totalwidth=\mytotalwidth]
        \begin{column}[t]{0.5\mycolumnwidth}
            \begin{itemize}
                \pause
            \item $\mathsf{0}^{(0)}$ \quad \alert{constant}
                \pause
            \item $\m{s}^{(1)}$ \quad \alert{unary function}
                \pause
            \item $\m{+}^{(2)}$ \quad \alert{binary function}
            \end{itemize}
        \end{column}
        \begin{column}[t]{0.5\mycolumnwidth}
            \begin{itemize}
                \pause
            \item $\m{ifThenElse}^{(3)}$ \quad \alert{ternary function}
                \pause
            \item $\m{f}^{(4)}$ \quad \alert{4-ary function}
            \end{itemize}
        \end{column}
        \end{columns}
    \end{example}
\end{frame}


\begin{frame}
    \frametitle{Terms}
    \begin{definition}
        let $\FF$ be sigunature and let $\VV$ be set of variables
        \begin{itemize}
            \item \alert{$\TT(\FF,\VV)$} is set of terms built from $\FF$ and $\VV$
            \item $t \in \TT(\FF,\VV)$ is \alert{term} if
                \begin{itemize}
                    \item $t = x$ and $x \in \VV$,
                    \item $t = f$, $f^n \in \FF$ and $n = 0$, or
                    \item $t = f(t_1, \dots, t_n)$, $f^n \in \FF$, and $t_1, \dots, t_n \in \TT(\FF,\VV)$ 
                \end{itemize}
        \end{itemize}
    \end{definition}
    \pause
    \begin{example}
        let $\FF = \{\m{0}^0, \m{s}^1, \m{+}^2\}$ and $\VV = \{x,y,\dots\}$
        \pause
        \vspace{-12pt}
        \begin{columns}[totalwidth=\mytotalwidth]
            \begin{column}[t]{0.5\mycolumnwidth}
                \begin{itemize}
                    \item $x$ and $\m{0}$ \quad \textcolor{green}{term}
                    \pause
                    \item $\m{s}(\m{0}) + x$ \quad \textcolor{green}{term}
                    \pause
                \end{itemize}
            \end{column}
            \begin{column}[t]{0.5\mycolumnwidth}
                \begin{itemize}
                    \item $x(y)$ and $\m{0}(x)$ \quad \textcolor{red}{not term}
                    \pause
                    \item $\m{s}(+)$ \quad \textcolor{red}{not term}
                \end{itemize}
            \end{column}
        \end{columns}
    \end{example}
\end{frame}


\begin{frame}
    \frametitle{Subterm Relation}
    \begin{definition}
        let $s,t \in \TT(\FF,\VV)$
        \begin{itemize}
            \item $s \unrhd t$ if
                \begin{itemize}
                    \item $s = t$ or
                    \item $s = f(s_1,\dots,s_n)$ and $s_i \unrhd t$ for some $1 \leq i \leq n$
                \end{itemize}
            \item $s \rhd t$ if $s \unrhd t$ and $s \neq t$
            %\item $t$ is \alert{subterm} of $s$ if $s \unrhd t$
            %\item $t$ is \alert{proper subterm} of $s$ if $s \rhd t$
        \end{itemize}
    \end{definition}
    \pause
    \begin{example}
        \begin{itemize}
            \pause
            \item $\m{s}(x) \unrhd x$ and $\m{s}(x) \rhd x$
            \pause
            \item $\m{s}(x) + \m{s}(y) \rhd \m{s}(x)$
            \pause
            \item $\m{s}(x + y) \rhd x$
        \end{itemize}
    \end{example}
\end{frame}

\begin{frame}
    \frametitle{Well-foundedness of $\rhd$}

    \begin{definition}
        $s \rhd_1 t$ if $s = f(s_1,\dots,s_n)$ and $s_i = t$ for some $1 \leq i \leq n$
    \end{definition}
    \pause
    \begin{lemma}
        $\rhd \subseteq \rhd_1^+$
    \end{lemma}
    \pause
    \begin{proof}
        Let $s \rhd t$. \pause
        Then we have $s = f(s_1,\dots,s_m)$ and $s_i \rhd t$ for some $i$.
        \pause
        We perform structural induction on $s_i$.
        \pause
        \begin{itemize}
            \item If $s_i = t$ then we have $s \rhd_1 t$. Hence $s \rhd_1^+ t$ holds.
            \pause
            \item If $s_i = g(s_{i1}, \dots, s_{in})$ and $s_{ij} \rhd t$ then by the I.H., we have $s_{ij} \rhd_1^+ t$.
        \end{itemize}
        Thus we obtain $s \rhd_1 s_{ij} \rhd_1^+ t$. Hence $s \rhd_1^+ t$ holds.
    \end{proof}
\end{frame}

\begin{frame}
    %\frametitle{Well-foundedness of $\rhd$}
    \begin{lemma}
        $\rhd$ is well-founded
    \end{lemma}
    \pause
    \begin{proof}
        Suppose $s_1 \rhd s_2 \rhd s_3 \rhd \cdots$.
        \pause
        Because $\rhd \subseteq \rhd_1^+$,
        we obtain $s_1 \rhd_1 \cdots \rhd_1 s_2 \rhd_1 \cdots \rhd_1 s_3 \rhd_1 \cdots$.
        \pause
        However, since the definition of terms, $\rhd_1$ is well-founded.
        \pause
        Hence we have contradiction.
    \end{proof}
\end{frame}

\begin{frame}
    \frametitle{$\unrhd$ is Partial Order}
    \begin{lemma}
        $\unrhd$ is partial order
    \end{lemma}
    \pause
    \begin{proof}
        We show that $\unrhd$ on terms is reflexive, transitive, and antisymmetric.

        \pause
        (reflexivity) For all term $t$, $t = t$ holds. Hence $t \unrhd t$ holds.

        \pause
        (transitivity) Let $s \unrhd t$ and $t \unrhd u$.
        \pause
        We perform structural induction on $s$.
        \begin{itemize}
            \pause
            \item If $s = t$ then trivially we have $s \unrhd u$.
            \pause
            \item If $s = f(s_1, \dots, s_n)$ and $s_i \unrhd t$ for some $1 \leq i \leq n$ then,
            by the I.H., we have $s_i \unrhd u$. Hence $s \unrhd u$ holds.
        \end{itemize}

        \pause
        (anti-symmetry) Suppose $s \unrhd t$ and $t \unrhd s$.
        \pause If $s \neq t$ we have $s \rhd t$ and $t \rhd s$.
        \pause It contradicts the well-foundedness of $\rhd$. Hence $s = t$ holds.
    \end{proof}
\end{frame}

\begin{frame}
    \frametitle{$\rhd$ is Strict Order}
    \begin{lemma}
        $\rhd$ is strict order
    \end{lemma}
    \pause
    \begin{proof}
        Firstly, irreflexivity of $\rhd$ is obtained by the definition of $\rhd$.
        \pause
        Secondly, we show that $\rhd$ is transitive. Suppose $s \rhd t$ and $t \rhd u$.
        \pause
        Then we have $s \unrhd t$, $t \unrhd u$, $s \neq t$, and $t \neq u$.
        \pause
        So we have $s \unrhd u$. If $s = u$ then it contradicts the well-foundedness of $\rhd$.
        \pause
        Thus we obtain $s \neq u$. Hence $s \rhd u$ holds.
    \end{proof}
    \begin{corollary}
        $\rhd$ is well-founded order
    \end{corollary}
\end{frame}

\end{document}
