\documentclass[autodetect-enginem]{article}

\usepackage{luatexja}
\usepackage{luatexja-fontspec}
\usepackage{luatexja-ruby}
\usepackage{lmodern, amsmath, amsthm, amssymb, proof}
\usepackage{tikz}
\usetikzlibrary{arrows, positioning}


\newcommand{\VV}{\mathcal{V}}
\newcommand{\TT}{\mathcal{T}}
\newcommand{\RR}{\mathcal{R}}
\newcommand{\FF}{\mathcal{F}}
\newcommand{\Var}{\mathcal{V}\mathrm{ar}}

\theoremstyle{plain}
\newtheorem{theorem}{Theorem}
\newtheorem*{theorem*}{Theorem}

\theoremstyle{definition}
\newtheorem{definition}{Definition}
\newtheorem*{definition*}{Definition}

\theoremstyle{definition}
\newtheorem{claim}{Claim}
\newtheorem*{claim*}{Claim}

% page number
% if you don't need page numbers, set "empty"
\pagestyle{empty}


\title{}
\author{}
\date{}

\begin{document}

\section*{A.8}
\subsection*{(a)}
    \begin{claim}
        $R^m \cdot R^n = R^{m+n}$ for all $m,n \in \mathbb{N}$.
    \end{claim}

    \begin{proof}
        We perfome induction on $n$.
        Suppose $(a,c) \in R^m \cdot R^n$.
        \begin{itemize}
            \item If $n = 0$ then $R^m \cdot R^0 = R^m \cdot \mathrm{id}_A = R^m$.
                    Thus $(a,c) \in R^{m+0}$.
            \item If $n = n' + 1$ then $(a,c) \in R^m \cdot R^{n'} \cdot R$.
                    So there exists some $b$ such that $(a,b) \in R^m \cdot R^{n'}$ and $(b,c) \in R$.
                    By I.H., we have $(a,b) \in R^{m+n'}$.
                    Thus $(a,c) \in R^{m+n'} \cdot R$.
                    Hence $(a,c) \in R^{m+n}$.
        \end{itemize}
        Conversely, suppose $(a,c) \in R^{m+n}$.
        \begin{itemize}
            \item If $n=0$ then $(a,c) \in R^m$. Thus $(a,c) \in R^m \cdot R^0$.
            \item If $n=n'+1$ then $(a,c) \in R^{m + n' + 1}$. So there exists some $b$ such that
                    $(a,b) \in R^{m+n'}$ and $(b,c) \in R$.
                    By I.H., we have $(a,b) \in R^{m} \cdot R^{n'}$.
                    Thus $(a,c) \in R^{m} \cdot R^{n'} \cdot R$.
                    Hence $(a,c) \in R^{m} \cdot R^{n}$.
        \end{itemize}
    \end{proof}


\subsection*{(b)}
    \begin{claim}
        $(R^n)^{-1} = (R^{-1})^n$ for all $n \in \mathbb{N}$.
    \end{claim}
    \begin{proof}
        We perfome induction on $n$.
        \begin{itemize}
            \item If $n = 0$: Suppose $(a,b) \in (R^0)^{-1}$.
                    Hence we have $(a,b) \in (R^{-1})^0$.
                    Conversely, suppose $(a,b) \in (R^{-1})^0$.
                    Hence we have $(a,b) \in (R^0)^{-1}$.
            \item If $n = n' + 1$: Suppose $(a,c) \in (R^{n'} \cdot R)^{-1}$.
                    Then $(c,a) \in R^{n'} \cdot R$.
                    So there exists some $b$ such that $(c,b) \in R^{n'}$ and $(b,a) \in R$.
                    Therefore $(b,c) \in (R^{n'})^{-1}$ and $(a,b) \in R^{-1}$.
                    By I.H., $(b,c) \in (R^{-1})^{n'}$.
                    Thus $(a,c) \in R^{-1} \cdot (R^{-1})^{n'}$.
                    Hence $(a,c) \in (R^{-1})^{n}$.
                    Conversely, suppose $(a,c) \in (R^{-1})^{n'} \cdot R^{-1}$.
                    Then there exists some $b$ such that $(a,b) \in (R^{-1})^{n'}$ and $(b,c) \in R^{-1}$.
                    By I.H., we have $(a,b) \in (R^{n'})^{-1}$.
                    So $(b,a) \in R^{n'}$ and $(c,b) \in R$.
                    Thus $(c,a) \in R \cdot R^{n'}$.
                    Hence $(a,c) \in (R^{n})^{-1}$.
            \end{itemize}
    \end{proof}

\subsection*{(c)}
    \begin{claim}
        $R^{m+n} \cdot (R^{-1})^n = R^m$ does not hold for all $n,m \in \mathbb{N}$.
    \end{claim}
    \begin{proof}
        Take
            \begin{itemize}
                \item $R =\{(0,1), (1,2)\}$,
                \item $m=1$, and $n=1$.
            \end{itemize}
        Then we have
            \begin{itemize}
                \item $R^2 = \{(0,2)\}$,
                \item $R^{-1} = \{(1,0), (2,1)\}$, and
                \item $R^2 \cdot R^{-1} = \{(0,1)\}$.
            \end{itemize}
        But $R^m = R^1 = R = \{(0,1), (0,2)\} \neq \{(0,2)\}$.
    \end{proof}

\subsection*{(d)}
    \begin{claim}
        $(R^*)^{-1} = (R^{-1})^{*}$
    \end{claim}
    \begin{proof}
        Firstly, we show $(\bigcup_{n \geq 0} R^n)^{-1} \subseteq \bigcup_{n \geq 0} (R^{-1})^n$.
        We perfome case analysis on $n$.
        \begin{itemize}
            \item If $n=0$ then $(a,b) \in \mathrm{id}_A$ then $(a,b) \in \mathrm{id}_A$.
            \item If $n>0$ then $(a,b) \in (\bigcup_{n \geq 0} R^{n} \cup R)^{-1}$.
                \begin{itemize}
                    \item If (a,b) \in 
                \end{itemize}
        \end{itemize}
    \end{proof}
\end{document}
