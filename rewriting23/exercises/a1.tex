\documentclass[autodetect-enginem]{article}

\usepackage{luatexja}
\usepackage{luatexja-fontspec}
\usepackage{luatexja-ruby}
\usepackage{lmodern, amsmath, amsthm, amssymb, proof}
\usepackage{tikz}
\usetikzlibrary{arrows, positioning}


\newcommand{\VV}{\mathcal{V}}
\newcommand{\TT}{\mathcal{T}}
\newcommand{\RR}{\mathcal{R}}
\newcommand{\FF}{\mathcal{F}}
\newcommand{\Var}{\mathcal{V}\mathrm{ar}}

\theoremstyle{plain}
\newtheorem{theorem}{Theorem}
\newtheorem*{theorem*}{Theorem}

\theoremstyle{definition}
\newtheorem{definition}{Definition}
\newtheorem*{definition*}{Definition}

\theoremstyle{definition}
\newtheorem*{claim}{Claim}
%\newtheorem*{claim*}{Claim}

% page number
% if you don't need page numbers, set "empty"
%\pagestyle{empty}


\title{}
\author{}
\date{}

\begin{document}

\section*{A.8}
\subsection*{(a)}
    \begin{claim}
        $R^m \cdot R^n = R^{m+n}$ for all $m,n \in \mathbb{N}$.
    \end{claim}

    \begin{proof}
        We perfome induction on $n$.
        Suppose $(a,c) \in R^m \cdot R^n$.
        \begin{itemize}
            \item If $n = 0$ then $R^m \cdot R^0 = R^m \cdot \mathrm{id}_A = R^m$.
                    Thus $(a,c) \in R^{m+0}$.
            \item If $n = n' + 1$ then $(a,c) \in R^m \cdot R^{n'} \cdot R$.
                    So there exists some $b$ such that $(a,b) \in R^m \cdot R^{n'}$ and $(b,c) \in R$.
                    By I.H., we have $(a,b) \in R^{m+n'}$.
                    Thus $(a,c) \in R^{m+n'} \cdot R$.
                    Hence $(a,c) \in R^{m+n}$.
        \end{itemize}
        Conversely, suppose $(a,c) \in R^{m+n}$.
        \begin{itemize}
            \item If $n=0$ then $(a,c) \in R^m$. Thus $(a,c) \in R^m \cdot R^0$.
            \item If $n=n'+1$ then $(a,c) \in R^{m + n' + 1}$. So there exists some $b$ such that
                    $(a,b) \in R^{m+n'}$ and $(b,c) \in R$.
                    By I.H., we have $(a,b) \in R^{m} \cdot R^{n'}$.
                    Thus $(a,c) \in R^{m} \cdot R^{n'} \cdot R$.
                    Hence $(a,c) \in R^{m} \cdot R^{n}$.
        \end{itemize}
    \end{proof}


\subsection*{(b)}
    \begin{claim}
        $(R^n)^{-1} = (R^{-1})^n$ for all $n \in \mathbb{N}$.
    \end{claim}
    \begin{proof}
        We perfome induction on $n$.
        \begin{itemize}
            \item If $n = 0$: Suppose $(a,b) \in (R^0)^{-1}$.
                    Hence we have $(a,b) \in (R^{-1})^0$.
                    Conversely, suppose $(a,b) \in (R^{-1})^0$.
                    Hence we have $(a,b) \in (R^0)^{-1}$.
            \item If $n = n' + 1$: Suppose $(a,c) \in (R^{n'} \cdot R)^{-1}$.
                    Then $(c,a) \in R^{n'} \cdot R$.
                    So there exists some $b$ such that $(c,b) \in R^{n'}$ and $(b,a) \in R$.
                    Therefore $(b,c) \in (R^{n'})^{-1}$ and $(a,b) \in R^{-1}$.
                    By I.H., $(b,c) \in (R^{-1})^{n'}$.
                    Thus $(a,c) \in R^{-1} \cdot (R^{-1})^{n'}$.
                    Hence $(a,c) \in (R^{-1})^{n}$.
                    Conversely, suppose $(a,c) \in (R^{-1})^{n'} \cdot R^{-1}$.
                    Then there exists some $b$ such that $(a,b) \in (R^{-1})^{n'}$ and $(b,c) \in R^{-1}$.
                    By I.H., we have $(a,b) \in (R^{n'})^{-1}$.
                    So $(b,a) \in R^{n'}$ and $(c,b) \in R$.
                    Thus $(c,a) \in R \cdot R^{n'}$.
                    Hence $(a,c) \in (R^{n})^{-1}$.
            \end{itemize}
    \end{proof}

\subsection*{(c)}
    \begin{claim}
        $R^{m+n} \cdot (R^{-1})^n = R^m$ does not hold for all $n,m \in \mathbb{N}$.
    \end{claim}
    \begin{proof}
        Take
            \begin{itemize}
                \item $R =\{(0,1), (1,2)\}$,
                \item $m=1$, and $n=1$.
            \end{itemize}
        Then we have
            \begin{itemize}
                \item $R^2 = \{(0,2)\}$,
                \item $R^{-1} = \{(1,0), (2,1)\}$, and
                \item $R^2 \cdot R^{-1} = \{(0,1)\}$.
            \end{itemize}
        However, $R^m = R^1 = R = \{(0,1), (0,2)\} \neq \{(0,2)\}$.
    \end{proof}

%\subsection*{(d)}
%    \begin{claim}
%        $(R^*)^{-1} = (R^{-1})^{*}$
%    \end{claim}
%    \begin{proof}
%        Firstly, we show $(\bigcup_{n \geq 0} R^n)^{-1} \subseteq \bigcup_{n \geq 0} (R^{-1})^n$.
%        We perfome case analysis on $n$.
%        \begin{itemize}
%            \item If $n=0$ then $(a,b) \in \mathrm{id}_A$ then $(a,b) \in \mathrm{id}_A$.
%            \item If $n>0$ then $(a,b) \in (\bigcup_{n \geq 0} R^{n} \cup R)^{-1}$.
%                \begin{itemize}
%                    \item If (a,b) \in 
%                \end{itemize}
%        \end{itemize}
%    \end{proof}

\newcommand{\RS}{R^{*}}
\newcommand{\RP}{R^{+}}
\newcommand{\RN}{R^{n}}
\newcommand{\NN}{\mathbb{N}}

\subsection*{(e)}
\begin{claim}
    $\RS \cdot R \subseteq \RP$
\end{claim}

\begin{proof}
    Let $(a,c) \in \RS \cdot R$.
    Then there exists some $b$ such that
    $(a,b) \in \RS$ and $(b,c) \in R$.
    Thus we have $(a,b) \in \RN$ for some $n \in \NN$.
    So $(a,c) \in \RN \cdot R$ and $\RN \cdot R = R^{n+1}$ holds.
    Hence $(a,c) \in \RP$.
\end{proof}

\begin{claim}
    $\RS \cdot \RS \subseteq \RS$
\end{claim}

\newcommand{\RM}{R^{m}}

\begin{proof}
    Let $(a,c) \in \RS \cdot \RS$.
    Then we have $(a,b) \in \RS$ and $(b,c) \in \RS$ for some $b$.
    So there exists some $n,m \in \NN$ such that $(a,b) \in \RN$ and $(b,c) \in \RM$.
    Therefore $(a,c) \in R^{n+m}$.
    Hence $(a,c) \in \RS$.
\end{proof}

\begin{claim}
    $\RS \cdot \RP \subseteq \RP$
\end{claim}

\begin{proof}
    Let $(a,c) \in \RS \cdot \RP$.
    Then we have $(a,b) \in \RS$ and $(b,c) \in \RP$ for some $b$.
    So there exists some $n,m \in \NN$ such that $(a,b) \in \RN$, $(b,c) \in \RM$, $n \geq 0$, and $m \geq 1$.
    Thus $(a,c) \in R^{n+m}$ and $n+m \geq 1$.
    Hence $(a,c) \in \RP$.
\end{proof}

\begin{claim}
    $\RP \cdot \RP \subseteq \RP$.
\end{claim}

\begin{proof}
    Let $(a,c) \in \RP \cdot \RP$.
    Then we have $(a,b) \in \RP$ and $(b,c) \in \RP$ for some $b$.
    So there exists some $n,m\in \NN$ such that $(a,b) \in \RN$, $(b,c) \in \RM$,
    $n \geq 1$, and $m \geq 1$.
    Thus $(a,c) \in R^{n+m}$ and $n+m \geq 1$.
    Hence $(a,c) \in \RP$.
\end{proof}

\subsection*{(f)}

\begin{claim}
    $\RP \subseteq \RS \cdot R$
\end{claim}

\begin{proof}
    Let $(a,c) \in \RP$.
    Then there exists some $n \in \NN$ such that $(a,b) \in \RN$ and $n \geq 1$.
    So we have $(a,c) \in R^{n-1} \cdot R$ and $n-1 \geq 0$.
    Thus we have $(a,b) \in R^{n-1}$ and $(b,c) \in R$ for some $b$.
    Hence $(a,c) \in \RS \cdot R$.
\end{proof}

\begin{claim}
    $\RS \subseteq \RS \cdot \RS$
\end{claim}

\begin{proof}
    Let $(a,c) \in \RS$.
    Then we have $(a,c) \in R^{n+m}$ and $n,m \geq 0$ for some $n,m \in \NN$.
    So we have $(a,b) \in \RN$ and $(b,c) \in \RM$ for some $b$.
    Thus $(a,b) \in \RS$ and $(b,c) \in \RS$.
    Hence $(a,c) \in \RS \cdot \RS$.
\end{proof}

\begin{claim}
    $\RP \subseteq \RS \cdot \RP$
\end{claim}

\begin{proof}
    Let $(a,c) \in \RP$.
    We already know that $\RP \subseteq \RS \cdot R$.
    So we have $(a,b) \in \RS$ and $(b,c) \in R$ for some $b$.
    Hence $(a,c) \in \RS \cdot \RP$.

\end{proof}

\begin{claim}
    $\RP \subseteq \RP \cdot \RP$ does not hold
\end{claim}

\begin{proof}
    Take
    \begin{itemize}
        \item $A = \{0,1,2\}$ and
        \item $R = \{(0,1),(1,2)\}$.
    \end{itemize}
    Then we have
    \begin{itemize}
        \item $\RP = \{(0,1),(1,2),(0,2)\}$ and
        \item $\RP \cdot \RP = \{(0,2)\}$.
    \end{itemize}
    So $(0,1) \in \RP$ but $(0,1) \notin \RP \cdot \RP$.
\end{proof}

\section*{A.9}
\subsection*{(c)}
\begin{claim}
    $(\RS)^{*} = (\RP)^{*} = (\RS)^{+} = \RS$
\end{claim}

\begin{proof}
    It is enoght to show that
    \begin{itemize}
        \item $(\RS)^{*} = \RS$,
        \item $(\RP)^{*} = \RS$, and
        \item $(\RS)^+ = \RS$.
    \end{itemize}

    Firstly, we show that $(\RS)^{n} \subseteq \RS$ for all $n \in \NN$ by induction on $n$.
        \begin{itemize}
            \item If $n = 0$ then trivially we have $(\RS)^n \subseteq \RS$.
            \item If $n > 0$ then we have $(a,c) \in (\RS)^{n-1} \cdot \RS$.
                    So there exists some $b$ such that $(a,b) \in (\RS)^{n-1}$
                    and $(b,c) \in \RS$. By I.H., we have $(a,b) \in \RS$.
                    Hence $(a,c) \in \RS$.
        \end{itemize}
        Conversely, it is obviously that $\RS \subseteq (\RS)^*$.
        So we obtain that $(\RS)^{*} = \RS$

    Secondly, we show that $(\RP)^n \subseteq \RS$ for all $n \geq 0$ by induction on $n$.
        \begin{itemize}
            \item If $n = 0$ then $(\RP)^0 \subseteq \RS$.
            \item If $n > 0$ then we have $(a,c) \in (\RP)^{n-1} \cdot \RP$.
                  So there exists some $b$ such that $(a,b) \in (\RP)^{n-1}$ and $(b,c) \in \RP$.
                  By I.H., we have $(a,b) \in \RS$. Hence $(a,c) \in \RS$.
        \end{itemize}
                  Additionally, we show taht $\RS \subseteq (\RP)^+$.$Let (a,b) \in \RS$.
                  Then $(a,b) \in \RN$ for some $n \geq 0$.
                  \begin{itemize}
                      \item If $n = 0$ then $(a,b) \in R^0 = \mathrm{id}_A$. So $(a,c) \in (\RP)^{*}$.
                      \item If $n > 0$ then we have $(a,b) \in R^n$ and $n \geq 1$. So $(a,c) \in (\RP)^{*}$.
                  \end{itemize} 
                 Hence we have $(\RP)^* = \RS$.

    Thirdly, we show that $(\RS)^n = \RS$ for all $n \geq 1$ by induction on $n$.
        \begin{itemize}
            \item If $n = 1$ then trivially we have $(\RS)^1 = \RS$.
            \item If $n > 1$ then we have $(\RS)^{n-1} \cdot \RS$. By I.H., $\RS \cdot \RS = \RS$.
        \end{itemize}
        So $(\RS)^+ = \RS$ holds.

    Finally, we conclude that the claim holds.
\end{proof}


\section*{A.10}
\subsection*{(a)}

\begin{claim}
    $(\RP)^+ = \RP$
\end{claim}

\begin{proof}
    It is enoght to show that $(\RP)^n \subseteq \RP$ for all $n \geq 1$.
    We perfome induction on $n$.
    If $n = 1$ then $(\RP)^1 = \RP \subseteq \RP$.
    If $n > 1$ then $(\RP)^n = (\RP)^{n-1} \cdot \RP$. Let $(a,c) \in (\RP)^{n-1} \cdot \RP$.
    So we have $(a,b) \in (\RP)^{n-1}$ and $(b,c) \in \RP$ for some $b$.
    By I.H., we have $(a,b) \in \RP$. Thus $(a,c) \in \RP \cdot \RP$.
    Hence $(a,c) \in \RP$.
\end{proof}

\newcommand{\SP}{S^{+}}
\begin{claim}
    $R \subseteq S \Rightarrow \RP \subseteq \SP$
\end{claim}

\begin{proof}
    Assume $R \subseteq S$.
    We show that $\RN \subseteq \SP$ for all $n \geq 1$ by induction on $n$.
    \begin{itemize}
        \item If $n = 1$ then $(a,c) \in R$ implies $(a,c) \in S$. So $(a,c) \in \SP$
        \item If $n > 1$, suppose $(a,c) \in R^{n}$. So we have $(a,b) \in R^{n-1}$ and $(b,c) \in R$ for some $b$.
              Then we have $(a,b) \in \SP$ by I.H. and  $(b,c) \in S$ by the assamption.
              Thus $(a,c) \in \SP \cdot \SP$.
              Hence $(a,c) \in \SP$.
    \end{itemize}
\end{proof}

\begin{claim}
    $(\cdot)^+$ is a closure operator.
\end{claim}

\begin{proof}
    Because $\RP$ is a transitive closure of $R$, $R \subseteq R$ holds.
    Additionally, we already show that idempotence and monotonicity of $(\cdot)^{+}$.
    Hence the claim holds.
\end{proof}

\section*{A.11}
\subsection*{(a)}

\begin{claim}
    $\RP = \RN$ for some $n$ dose not hold for arbitrary relation $R$ on a finite set $A$.
\end{claim}

\begin{proof}
    Take
    \begin{itemize}
        \item $A = \{0,1,2\}$ and
        \item $R = \{(0,1), (1,2)\}$.
    \end{itemize}

    Then we have
    \begin{itemize}
        \item $\RP = \{(0,1),(1,2),(0,2)\}$,
        \item $R^0 = \mathrm{id}_A = \{(0,0), (1,1), (2,2)\}$,
        \item $R^1 = \{(0,1), (1,2)\}$,
        \item $R^2 = \{(0,2)\}$, and
        \item $R^n = \emptyset$ with $n \geq 3$.
    \end{itemize}
    So we have $\RP \neq R^n$ for all $n$.
    Hence the claim holds.
\end{proof}

\section*{A.13}
\subsection*{(1)}
\newcommand{\lex}{(>_A, >_B)_{\mathsf{lex}}}
\begin{claim}
    $(>_A, >_B)_{\mathsf{lex}}$ is a strict order if $>_A$ and $>_B$ are strict orders.
\end{claim}

\begin{proof}
    Assume $>_A$ and $>_B$ are strict orders on $A$ and $B$ respectively.
    
    Firstly, we show that $\lex$ is irreflexive.
    Let $((a_1, b_1), (a_2,b_2)) \in \lex$.
    Then we have $a_1 >_A a_2$ or $a_1 = a_2$ and $b_1 >_B b_2$.
        \begin{itemize}
            \item If $a_1 >_A a_2$ then since $>_A$ is a strict order, $a_2 >_A a_1$ does not hold.
            \item If $a_1 = a_2$ and $b_1 >_B b_2$ then $b_2 >_B b_1$ does not hold.
        \end{itemize}
        So $((a_2, b_2), (a_1, b_1)) \notin \lex$. 
   
    Secondly we show that $\lex$ is transitive.
    Let $((a_1, b_1) (a_2, b_2)) \in \lex$ and $((a_2, b_2), (a_3, b_3)) \in \lex$.
    \begin{itemize}
        \item If $a_1 >_A a_2$ then we have two cases:
            \begin{itemize}
                \item If $a_2 >_A a_3$ then we have $a_1 >_A a_3$.
                \item If $a_2 = a_3$ and $b_2 >_B b_3$ then $a_1 >_A a_3$.
            \end{itemize}
        \item If $a_1 = a_2$ and $b_1 >_B b_2$ then also we have two cases:
            \begin{itemize}
                \item If $a_2 >_A a_3$ then $a_1 >_A a_3$.
                \item If $a_2 = a_3$ and $b_2 >_B b_3$ then we have $a_1 = a_3$ and $b_1 >_B b_3$.
            \end{itemize}
    \end{itemize}
    Hence $\lex$ is transitive.
    Finally, we conclude that $\lex$ is a strict order.
    
\end{proof}

\end{document}
