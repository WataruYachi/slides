\documentclass[autodetect-engine]{article}

\usepackage{luatexja}
\usepackage{luatexja-fontspec}
\usepackage{luatexja-ruby}
\usepackage{lmodern, amsmath, amsthm, amssymb, proof}
\usepackage{tikz}
\usetikzlibrary{arrows, positioning}

%\usepackage[margin=20truemm]{geometry}
\usepackage{here}

\newcommand{\VV}{\mathcal{V}}
\newcommand{\TT}{\mathcal{T}}
\newcommand{\RR}{\mathcal{R}}
\newcommand{\FF}{\mathcal{F}}
\newcommand{\Var}{\mathcal{V}\mathrm{ar}}

\theoremstyle{plain}
\newtheorem{theorem}{Theorem}
\newtheorem*{theorem*}{Theorem}

\theoremstyle{definition}
\newtheorem{definition}{Definition}
\newtheorem*{definition*}{Definition}

\theoremstyle{definition}
\newtheorem*{claim}{Claim}
%\newtheorem*{claim*}{Claim}

% page number
% if you don't need page numbers, set "empty"
\pagestyle{empty}


\title{}
\author{}
\date{}

\newcommand{\m}[1]{\mathsf{#1}}
\newcommand{\Fun}{\FF\mathsf{un}}

\begin{document}

\section*{2.1 (b)}

Let $t = (\m{s}(\m{0}) + x) + \m{s}(\m{s}(\m{0}))$.
Then $t$ is linear because the variable $x$ occurs once.

\section*{2.3 (a)}

\begin{definition*}
    Let $t$ be a term and let $a$ be a function symbol or variable.
    \[
        |t|_a = \begin{cases}
            1 & \text{if $t = a$ and $a$ is a variable}\\
            \sum\limits_{i = 1}^n |t_i|_a & \text{if $t = f(t_1,\dots,t_n)$ and $f \neq a$}\\
            1 + \sum\limits_{i = 1}^n |t_i|_a & \text{if $t = f(t_1,\dots,t_n)$ and $f = a$}\\
        \end{cases}
    \]
\end{definition*}

\begin{claim}
    If $a \notin \Fun(t) \cup \Var(t)$ then $|t|_a = 0$.
\end{claim}

\begin{proof}
    Let $t$ be a term and let $a$ be a function symbol or variable.
    Suppose that $a \notin \Fun(t) \cup \Var(t)$.
    We perform structural induction on $t$.
    \begin{itemize}
        \item If $t$ is a variable or constant then trivially we have $|t|_a = 0$. 
        \item If $t = f(t_1,\dots,t_n)$ then by the I.H., $|t_i|_a = 0$ for all $0 \leq i \leq n$ holds.
              By the assumption, $f \neq a$. Hence $|t|_a = 0$ holds.
    \end{itemize}
\end{proof}

\newcommand{\Pos}{\mathcal{P}\mathsf{os}}

\section*{2.4}

\begin{claim}
    For all terms $t$ and $s$,
    if there exists some $p \in \Pos(t)$ such that $p \neq \varepsilon$ and $t|_p = s$ then
    there exists some non-empty context $C$ such that $t = C[s]$.
\end{claim}

\begin{proof}
    We perform structural induction on $t$.
    \begin{itemize}
        \item If $t$ is a variable or constant then vacuously the claim holds.
        \item If $t = f(t_1,\dots, t_n$) then take $C = t[\,]_p$.
        Since $p \neq \varepsilon$, we have $p = iq$ for some $i,q$ and $C \neq \Box$.
        Moreover, we have $t[\,]_p = f(t_1, \dots, t_i[\,]_q,\dots, t_n)$ and $s = t|_p = t_i|_q$.
        We distinguish two cases.
        \begin{itemize}
            \item If $q=\varepsilon$ then we have $t[\,]_p = f(t_1,\dots,\Box,t_n)$ and $s = t_i$.
                Thus $C[s] = (t[\,]_p)[s] = f(t_1,\dots,\Box[s],\dots,t_n) = f(t_1,\dots,t_i,\dots,t_n) = t$.
                Hence the claim holds.
            \item If $q \neq \varepsilon$ we have $s = t_i|_q$. By the I.H., we have $(t_i[\,]_q)[s] = t_i$.
                Thus $C[s] = (t[\,]_p)[s] = f(t_1,\dots,(t_i[\,]_q)[s],\dots,t_n) = f(t_1,\dots,t_i,\dots,t_n) = t$.
                Hence the claim holds.
        \end{itemize}
    \end{itemize}
\end{proof}

\section*{2.6 (a)}

Let $t = (\m{0} + \m{s}(\m{0})) + (\m{s}(\m{s}(\m{0})) + (\m{0} + \m{0}))$.
Then $t|_{21} = \m{s}(\m{s}(\m{0}))$.

\section*{2.7 (b)}

\begin{claim}
    $|t| = \|t\| + |\Pos_\VV(t)|$
\end{claim}

\begin{proof}
    We perform structural induction on $t$.
    \begin{itemize}
        \item If $t$ is a variable $x$ then we have $|x| = 1$, $\|x\| = 0$, and $|\Pos_\VV(x)| = 1$.
            Hence $|x| = 1 = \|x\| + |\Pos_\VV(x)| holds$.
        \item If $t$ is a constant $f$ then we have $|f| = 1$, $\|f\| = 1$, and $|\Pos_\VV(f)| = 0$.
            Hence $|f| = 1 = \|f\| + |\Pos_\VV(f)|$ holds.
        \item If $t = f(t_1,\dots,t_n)$ then we have $|t| = 1 + |t_1| + \cdots + |t_n|$,
            $\|t\| = 1 + \|t_1\| + \cdots + \|t_n\|$, and $|\Pos_\VV(t)| = |\Pos_\VV(t_1)| + \cdots + |\Pos_\VV(t_n)|$.
            By the I.H., for all $1 \leq i \leq n$, $|t_i| = \|t_i\| + |\Pos_\VV(t_i)|$ holds.
            Thus we have $|t| = 1 + \|t_1\| + \cdots \|t_n\| + |\Pos_\VV(t_1)| + \cdots + |\Pos_\VV(t_n)|$.
            Hence $|t| = \|t\| +|\Pos_\VV(t)|$ holds. 
    \end{itemize}
\end{proof}

\begin{claim}
    $t$ is ground $\Leftrightarrow$ $|t| = \|t\|$
\end{claim}

\begin{proof}
    ($\Rightarrow$-direction) Let $t$ be a ground term.
    We perform structural induction on $t$.
    \begin{itemize}
        \item If $t$ is constant then $|t| = 1 = \|t\|$ holds.
        \item If $t = f(t_1,\dots,f_n)$ then we have $|t| = 1 + |t_1| + \cdots + |t_n|$ and $\|t\| = 1 + \|t_1\| + \cdots + \|t_n\|$.
        Since $t$ is ground, for all $1 \leq i \leq n$, $t_i$ is ground.
        By the I.H., we have $|t_i| = \|t_i\|$.
        Hence $|t| = \|t_1\|$ holds.
    \end{itemize}

    ($\Leftarrow$-direction)
    Suppose $|t| = \|t\|$.
    Assume that $t$ is not ground.
    Then $|\Pos_\VV(t)| > 0$.
    Since $|t| = \|t\| + |\Pos_\VV(t)|$, $|t| \neq \|t\|$.
    Hence we have contradiction.
\end{proof}



\section*{2.9 (b)}

\begin{claim}
    For all position $p$ and $q$, one of $p=q$, $p < q$, $p \parallel q$, and $q < p$ holds.
\end{claim}

\begin{proof}
    Assume that for some $p$ and $q$,
    $p\neq q$, $p \nless q$, $p \nparallel q$, and $q \nless p$ holds.
    Since $p \nparallel q$, we have $p \leqslant q$ or $q \leqslant p$.
    \begin{itemize}
        \item If $p \leqslant q$, since $p \neq q$, we obtain $p < q$.
            However we have $p \nless q$. Hence we have contradiction.
        \item If $q \leqslant p$, similarly we have contradiction.
    \end{itemize}
\end{proof}

\section*{2.11}

\begin{claim}
    For all positions $p$, $q$, $r$, and $s$,
    if $p \parallel q$ then $pr \parallel qs$ and $rq \parallel sq$.
\end{claim}

\begin{proof}
    Suppose $p \parallel q$. Firstly, we show that $pr \parallel qs$ for all $r$ and $s$.
    Assume that $pr \leq qs$ or $qs \leq pr$.
    
\end{proof}

%\begin{claim}
%    Let $p$ and $q$ be a position.
%    If $p \parallel q$ then $pr \parallel qr$ and $qr \parallel qr$ holds for all position $r$.
%\end{claim}

%\begin{proof}
%    Let $p \parallel q$ and let $r$ be a position.
%    ($pr \parallel qr$)
%    Suppose $pr \leqslant qr$ or $rp \leqslant rq$.
%    \begin{itemize}
%        \item If $pr \leqslant qr$ then there exists some $r'$ such that $prr' = qr$.
%            By eliminating $r'$, we obtain $pr' = q$. Thus $p \nparallel q$. We have contradiction.
%        \item Otherwise, similarly we have contradiction.
%    \end{itemize}
%    Additionally, we can show that $rp \parallel rq$ in the same way.
%\end{proof}

\section*{2.12 (d)}

\begin{claim}
    Let $P$ be a finite set of positions.
    Then ${>} \cup {>_1}$ on $P$ is a well-order.
\end{claim}

\newcommand{\rr}{> \cup >_1}

\begin{proof}
    By previous question, $> \cup >_1$ is total order.
    So, we show that $> \cup >_1$ is well-founded on every finite set of position $P$.
    
    Assume that there exists an infinite descending sequence $p_1 \rr p_2 \rr p_3 \rr \cdots$ of element $P$.
    Since $P$ is finite, we have $p_i = p_j$ for some $i$ and $j$ such that $i < j$.
    Because $\rr$ is transitive, we have $p_i > p_j$.
    Hence we have contradiction.
\end{proof}

\section*{2.14 (a)}

\begin{claim}
    Let $s$ and $t$ be a term.
    For all positions $p,q \in \Pos(s)$, if $p \parallel q$ then $(s[t]_p)|_q = s|_q$. 
\end{claim}

\begin{proof}
    Let $s$ and $t$ be a term and $p,q \in \Pos(s)$.
    We perform structural induction on $s$.
    \begin{itemize}
        \item If $s$ is a variable or constant then the claim holds vacuously.
        \item If $s = f(s_1,\dots,s_n)$ then suppose $p \parallel q$.
        We distinguish two cases.
            \begin{itemize}
                \item If $p = ip'$ and $q = iq'$ then
                    we have
                        \[
                            (s[t]_p)|_q = f(s_1, \dots, s_i[t]_{p'}, \dots, s_n)|_{iq'} = (s_i[t]_p')|_{q'}
                        \]
                    and \[
                            s|_q = s_i|_{q'}.
                        \]
                    Since $p' \parallel q'$ and the I.H., we obtain $(s_i[t]_{p'})|_{q'} = s_i|_{q'}$.
                    Hence we have $(s[t]_p)|_q = s|_q$.
                \item If $p = ip'$, $q = jq'$ and $i \neq j$ then
                    we have
                    \[(s[t]_p)|_q = f(s_1, \dots, s_i[t]_{p'}, \dots, s_j, \dots, s_n)|_{jq'}
                    = s_j|_{q'}\]
                    and \[
                        s|_q = s_j|_{q'}.
                    \]
                    Hence $(s[t]_p)|_q = s|_q$ holds.
            \end{itemize}
    \end{itemize}
\end{proof}

\section*{2.15}
Let $t = x + (y + (x + y))$ and $\sigma = \{x \mapsto \m{0}+z, y \mapsto \m{s}(\m{0}), z \mapsto x + x\}$.
Then $t\sigma = (\m{0} + z) + (\m{s}(0) + ((\m{0} + z) + \m{s}(\m{0})))$ and
$\mathcal{D}\mathsf{om}(\sigma) = \{x,y,z\}$.

\end{document}
