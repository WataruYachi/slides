\documentclass[autodetect-enginem]{article}

\usepackage{luatexja}
\usepackage{luatexja-fontspec}
\usepackage{luatexja-ruby}
\usepackage{lmodern, amsmath, amsthm, amssymb, proof}
\usepackage{tikz}
\usetikzlibrary{arrows, positioning}


\newcommand{\VV}{\mathcal{V}}
\newcommand{\TT}{\mathcal{T}}
\newcommand{\RR}{\mathcal{R}}
\newcommand{\FF}{\mathcal{F}}
\newcommand{\Var}{\mathcal{V}\mathrm{ar}}

\theoremstyle{plain}
\newtheorem{theorem}{Theorem}
\newtheorem*{theorem*}{Theorem}
\newtheorem*{lemma*}{Lemma}

\theoremstyle{definition}
\newtheorem{definition}{Definition}
\newtheorem*{definition*}{Definition}

\newtheorem{claim}{Claim}
\newtheorem*{claim*}{Claim}

% page number
% if you don't need page numbers, set "empty"
\pagestyle{empty}

\newcommand{\lto}{\reflectbox{$\rightarrow^*$}}

\title{}
\author{}
\date{}

\begin{document}

\section*{1.7 (b)}

Eliminate $e \rightarrow b$. Then the ARS terminates.

\section*{1.9 (a)}

\begin{claim*}
    Every normal form is complete.
\end{claim*}

\begin{proof}
    Let $a$ be a normal form of ARS $\mathcal{A}$.
    We show that $a$ is terminating and confluent.
    Firstly, it is obvious that $a$ is terminating because there is no $b$ such that $a \rightarrow b$.
    Secondly we show that $a$ is confluent.
    Assume $b \;\reflectbox{$\rightarrow^*$}\; a \rightarrow^* c$ for some $b$ and $c$.
    Since $a$ is a normal form, we have $a=b$ and $a=c$. Thus $b \rightarrow^* a \;\reflectbox{$\rightarrow^*$} \; c$.
    Hence $a$ is complete.
\end{proof}

\section*{1.10 (a)}

\begin{claim*}
    The ARS $\mathcal{A}$ of Exercise 1.5 is terminating.
\end{claim*}

\begin{proof}
    Assume $a_0 \to a_1 \to a_2 \to \cdots$.
    By the definition of $\to$, for all $i \in \mathbb{N}$, $a_i > a_{i+1}$ holds.
    So we have $a_0 > a_1 > a_2 > \cdots$.
    However $>$ on $\mathbb{N}$ has the least element.
    Hence we have contradiction.
\end{proof}


\section*{1.11 (a)}
\begin{claim*}
    Let $\mathcal{A} = \langle A, \to \rangle$ be an ARS. If $a \in A$ is terminating then
    $b \in \{b \mid a \rightarrow^* b\}$ is terminating.
\end{claim*}

\begin{proof}
    Let $a$ be a terminating element of $A$ and $a \to^* b_0$.
    Assume $b_0 \to b_1 \to b_2 \to \cdots$.
    Since $a \to^* b_0$, we have $a \to^* b_0 \to b_1 \to b_2 \to \cdots$.
    Then $a$ is not terminating, which is contradiction.
\end{proof}

\section*{1.12}

\newcommand{\conv}{\leftrightarrow}
\newcommand{\join}{\downarrow}
\newcommand{\ito}{\leftarrow}
\begin{lemma*}
    For every ARS $\mathcal{A} = \langle A, \to \rangle$,
    $\conv^* \; \subseteq \; \downarrow \cup \conv^* \cdot \leftarrow \cdot \rightarrow \cdot \conv^*$ holds.
\end{lemma*}

\begin{proof}
    Let $a \conv^n e$.
    We perform induction on $n$.
    \begin{itemize}
        \item If $n = 0$ then we have $a = e$. So $a \downarrow e$.
            Thus the claim holds. %$a \downarrow \cup \; \conv^* \cdot \leftarrow \cdot \rightarrow \cdot \conv^* e$.
        \item If $n > 0$ then we have $a \conv^{n-1} d \conv e$ for some $d$.
            By the I.H., we have $a \downarrow \cup \conv^* \cdot \leftarrow \cdot \rightarrow \cdot \conv^* d$.
            We perform case analysis.
            \begin{itemize}
                \item If $a \join d \ito e$, we have $a \join e$.
                \item If $a \join d \to e$ namely $a \to^* b \; \lto \; d \to e$, we have two cases.
                    \begin{itemize}
                        \item If $a \to^* b = d \to e$, we have $a \join e$.
                        \item If $a \to^* b \; \lto \; c \ito d \to e$,
                            we have $a \conv^* \cdot \ito \cdot \to \cdot \conv^* e$.
                    \end{itemize}
                \item If $a \conv^* \cdot \ito \cdot \to \cdot \conv^* d \conv e$, we have
                    $a \conv^* \cdot \ito \cdot \to \cdot \conv^* e$.
            \end{itemize}
    \end{itemize}
\end{proof}

\section*{1.13 (c)}

\begin{claim*}
    Let $\mathcal{A} = \langle A, \to \rangle$ be an ARS.
    For all $a \in A$, if $\mathcal{A}$ is weakly normalizing and $a$ has the unique normal form then $a is confluent$.
\end{claim*}

\begin{proof}
    Assume $c \;\lto\; a \to^* b$. Since $\mathcal{A}$ is weakly normalizing, there exists some
    $b'$ and $c'$ such that $b \to^! b'$ and $c \to^! c'$.
    Because $a$ has unique normal form property, we have $b' = c'$.
    Thus $b \join c$. Hence $a$ is confluent.    
\end{proof}

\section*{1.14 (b)}
Let $A = \langle A, \to \rangle$ be an ARS with the diamond property.

\begin{claim*}
    For all $a \in A$, $a$ is terminating if and only if $a \in \mathsf{NF}(\mathcal{A})$.
\end{claim*}

\begin{proof}
    To prove the claim, we show that for all $a \in A$, $a$ is not terminating if and only if $a \notin \mathsf{NF}(\mathcal{A})$.

    (Only if-direction) Assume $a = a_0 \to a_1 \to a_2 \to \cdots$.
    It is obvious that $a \notin \mathsf{NF}(\mathcal{A})$ because we have $a = a_0 \to a_1$.
    
    (If-direction) Let $a = a_0$. Suppose $a_0 \to a_1$ for some $a_1$.
    Since $\mathcal{A}$ has the diamond property,
    we have $a_1 \to a_2 \ito a_1$ for some $a_2$. %Then we have $a_0 \to a_1 \to a_2$.
    In addition, we can construct $a_3, a_4, a_5, \dots$ in the same way.
    Hence we obtain the infinite rewrite sequence $a_0 \to a_1 \to a_2 \to a_3 \to \cdots$.
\end{proof}

\begin{claim*}
    $\mathcal{A}$ is terminating if and only if $\to \; = \emptyset$.
\end{claim*}

\begin{proof}
    If-direction is trivial.

    (Only if-direction)
    Assume $\mathcal{A}$ is terminating.
    Suppose $(a,b) \in \;\to$ for some $a,b \in A$.
    According to the previous claim, $a \to b$ implies $a$ is not terminating.
    So $\mathcal{A}$ is not terminating. We have contradiction.
    Hence $\to\; = \emptyset$.
    

\end{proof}

\section*{1.15 (c)}

\begin{claim*}
    The ARS is not terminating.
\end{claim*}

\begin{proof}
    Take (1,1).
    Then we have $(1,1) \to (1,1) \to (1,1) \to \cdots$.
\end{proof}


\end{document}
