% 4up version
%\documentclass[handout, 12pt, aspectratio=169]{beamer}

\documentclass[12pt,aspectratio=169]{beamer}

\usefonttheme{professionalfonts}

\mode<handout>
{
\usepackage{pgfpages}
\pgfpagesuselayout{4 on 1}[a4paper,landscape]
}

\usepackage[english]{babel}
\usepackage{luatexja}
\usepackage{luatexja-fontspec}


% for display code
%\usepackage{minted}
%\newfontfamily\hasklig{hasklig}[NFSSFamily=haskligFamily]
%\setminted[haskell]{fontfamily=haskligFamily,
%mathescape,
%       numbersep=5pt
%}

% for proof tree
%\usepackage{bcprules}

%\usepackage[T1]{fontenc}

%for lualatex
\usepackage{fontspec}
\setsansfont{CMU Sans Serif}%{Arial}
\setmainfont{CMU Serif}%{Times New Roman}
\setmonofont{CMU Typewriter Text}%{Consolas}

\usepackage{lmodern,amsmath,amssymb,proof}
\usepackage{tikz}
\usetikzlibrary{arrows,positioning}



% color scheme
\definecolor{green}{rgb}{0.0,0.5,0.0}
\definecolor{blue}{rgb}{0.0,0.0,0.7}
\definecolor{red}{rgb}{0.8,0.0,0.0}
\definecolor{lightred}{rgb}{1.0,0.97,0.97}
\definecolor{lightblue}{rgb}{0.95,0.95,1.0}
\definecolor{darkblue}{rgb}{0.3,0.3,0.5}
\definecolor{darkred}{rgb}{0.6,0.0,0.0}
\definecolor{darkorange}{rgb}{0.6,0.2,0.1}
\definecolor{darkgreen}{rgb}{0,0.3,0}
%
\newcommand{\IMAGE}[2]{\pgfdeclareimage[#1]{#2}{#2}\pgfuseimage{#2}}

\setbeamerfont{title}{series=\bfseries,size=\Large}
\setbeamercolor{title}{fg=red}
\setbeamerfont{subtitle}{series=\bfseries,size=\LARGE}
\setbeamercolor{subtitle}{fg=red}
\setbeamerfont{author}{series=\bfseries,size=\large}
\setbeamercolor{author}{fg=blue}
\setbeamerfont{institute}{series=\bfseries,size=\large}
\setbeamercolor{institute}{fg=green}

\setbeamertemplate{blocks}[rounded]
\setbeamerfont{block title}{series=\bfseries,family=\sffamily,size=\small\strut}
\setbeamercolor{block title}{bg=darkblue,fg=white}
\setbeamercolor{block body}{bg=blue!4!white}
\setbeamercolor{block title example}{bg=white,fg=darkgreen}
\setbeamercolor{block body example}{bg=white}
\setbeamercolor{block title alerted}{bg=darkred,fg=white}
\setbeamercolor{block body alerted}{bg=red!4!white}
\setbeamercolor{structure}{fg=darkred}
% no fading effect
\makeatletter
\pgfdeclareverticalshading[lower.bg,upper.bg]{bmb@transition}{200cm}{%
  color(0pt)=(lower.bg); color(4pt)=(lower.bg); color(4pt)=(upper.bg)}
\makeatother

% frametitle
\useframetitletemplate{
\begin{centering}
\centerline{\large\bfseries\color{darkblue}\insertframetitle}
\end{centering}
}

% footer  (title  page/pages)
\setbeamertemplate{navigation symbols}{}
\useheadtemplate{\vbox{\vskip8pt}}
\usefoottemplate{\vbox{\vskip2pt\inserttitle\hfil\insertframenumber/\inserttotalframenumber\vskip5pt}}

% enumerate/itemize environment
\newcommand*\tikzboxed[1]{\tikz[baseline=(c.base)]{%
\node[thick,shape=rectangle,draw,inner sep=2pt] (c) {#1};}}
\setbeamertemplate{items}[square]
\setbeamertemplate{enumerate item}{\tikzboxed{\footnotesize\insertenumlabel}}
\setlength{\itemsep}{1ex}

% claim block
\makeatletter
\def\th@claim{%
\normalfont %body font
\setbeamercolor{block title example}{bg=darkorange, fg=white}
\setbeamercolor{block body example}{bg=darkorange!10, fg=black}
\setbeamercolor{example text}{fg=darkorange}
\def\inserttheoremblockenv{exampleblock}
}
\makeatother
\theoremstyle{claim}
\newtheorem{claim}[theorem]{Claim}%

\newcommand{\m}[1]{\mathsf{#1}}
\newcommand{\mi}[1]{\mathit{#1}}
\newcommand{\md}[1]{\mathtt{\textcolor{blue}{#1}}}
\newcommand{\seq}[2][n]{{#2_1},\dots,{#2_{#1}}}
%
\newcommand{\FF}{\mathcal{F}}
\newcommand{\RR}{\mathcal{R}}
\newcommand{\VV}{\mathcal{V}}
\newcommand{\TT}{\mathcal{T}}
\newcommand{\Var}{\mathcal{V}\m{ar}}
%
\newcommand{\app}{\circ}

\title{Term Rewriting}
\subtitle{Solutions for Exercises in Appendix 1}
\author{Wataru Yachi}
\institute{JAIST}
\date{Nov 2, 2023}

\begin{document}

\maketitle
\newcommand{\R}[3]{#1 \; #2 \; #3}
\newcommand{\sn}[1]{>_1^{#1}}

\begin{frame}
    \frametitle{What is $ran(\sn{n})$?}
    \begin{example}
        \newcommand{\idn}{\; \mathrm{id}_\mathbb{N} \;}
        \begin{itemize}
            \setlength{\itemsep}{+5mm}
            \pause
            \item $\sn{0}\; = \mathrm{id}_\mathbb{N} = \{0 \idn 0, 1 \idn 1, 2 \idn 2, 3 \idn 3, \cdots\}$
            \pause \quad $\therefore ran(\idn) = \{0,1,2,3,\cdots\}$
            \pause
            \item $\sn{1}\; = \; >_1 \cdot \;\mathrm{id}_\mathbb{N} = \{0 >_1 1, 1 >_1 2, 2 >_1 3, 3 >_1 4, cdots\}$
            \pause
            \quad $\therefore ran(\sn{1}) = \{1,2,3,4,\cdots\}$
            \pause
            \item $\sn{2}\; = \; >_1 \cdot >_1 = \{ 0 \sn{2} 2, 1 \sn{2} 3, 2 \sn{2} 4, 3 \sn{2} 5, \cdots \}$
            \pause \quad $\therefore ran(\sn{2}) = \{2, 3, 4, 5, \cdots \}$
            \pause
            \item $\sn{3}\; = \; \sn{2} \cdot >_1  = \{ 0 \sn{3} 3, 1 \sn{3} 4, 2 \sn{3} 5, 3 \sn{3} 6, \cdots \}$
            \pause
            $\therefore ran(\sn{3}) = \{3,4,5,6,\cdots\}$
            %\pause
            %\item $\sn{4}\; = \;\sn{3} \cdot >_1 = \{ 0 \sn{4} 4, 1 \sn{4} 5, 2 \sn{4} 6, 3 \sn{4} 7, \cdots \}$
            %\pause
            %$\therefore ran(\sn{4}) = \{4,5,6,7\cdots\}$
        \end{itemize}
    \end{example}
\end{frame}

\begin{frame}
    \begin{claim}
        $ran(\sn{n}) = \{n, n+1, n+2, n+3, \cdots\}$
    \end{claim}
    \pause
    \begin{proof}
        \pause
        We perfome induction on $n$.
        \pause
        \begin{itemize}
            \pause
            \item If $n=0$ then $\sn{0} \; = \mathrm{id}_{\mathbb{N}}$.
            \pause
            So $ran(\sn{0}) = \mathbb{N} = \{0,1,2,3,\cdots \}$.
            \pause
            \item If $n=n'+1$ then by I.H., $ran(\sn{n'}) = \{n', n'+1, n'+2, n'+3, \cdots \}$.\\
                \pause
                    So $ran(\sn{n'} \cdot >_1) = \{n'+1, n'+1+1, n'+2+1, n'+3+1, \cdots \}$.\\
                    \pause
                    Hence $ran(\sn{n}) = \{n, n+1, n+2, n+3, \cdots\}$.
        \end{itemize}
    \end{proof}
\end{frame}

\begin{frame}
    \begin{claim}[A.3(a)]
      $(R^{-1})^{-1} = R$
    \end{claim}
    \begin{proof}
      %Let $R$ be a relation on a set $A$.
      \begin{itemize}
            \pause
            \item Suppose $(a,b) \in (R^{-1})^{-1}$.\pause
                  Then we have $(b,a) \in R^{-1}$.\pause
                  Thus $(a,b) \in R$. Hence $(R^{-1})^{-1} \subseteq R$.
            \pause
            \item Suppose $(a,b) \in R$.\pause
                  Then $(b,a) \in R^{-1}$. Thus $(a,b) \in (R^{-1})^{-1}$.\pause
                  Hence  $R \subseteq (R^{-1})^{-1}$.
      \end{itemize}
    \end{proof}
\end{frame}

\begin{frame}
    \begin{claim}[A.4(b)]
        $(R_1 \cup R_2)^{-1} = R_1^{-1} \cup R_2^{-1}$
    \end{claim}
    \begin{proof}
        \pause
        Suppose $(a,b) \in (R_1 \cup R_2)^{-1}$.\pause
        Then $(b,a) \in R_1$ or $(b,a) \in R_2$.\pause
        \begin{itemize}
          \item If $(b,a) \in R_1$ then $(a,b) \in R_1^{-1} or (a,b) \in R_2^{-1}$. \pause
          \item If $(b,a) \in R_2$ then $(a,b) \in R_2^{-1} or (a,b) \in R_2^{-1}$. \pause
        \end{itemize}
        Thus $(a,b) \in R_1^{-1} \cup R_2^{-1}$.\pause
        Conversely, suppose $(a,b) \in R_1^{-1} \cup R_2^{-1}$. \pause
        \begin{itemize}
          \item If $(a,b) \in R_1^{-1}$ then $(b,a) \in R_1$ or $(b,a) \in R_2$.\pause
          \item If $(a,b) \in R_2^{-1}$ then $(b,a) \in R_2$ or $(b,a) \in R_1$.\pause
        \end{itemize}
        So $(b,a) \in R_1 \cup R_2$. Hence $(a,b) \in (R_1 \cup R_2)^{-1}$
    \end{proof}
\end{frame}

\begin{frame}
    \begin{claim}[A.5(a)]
        $R \cdot (S \cup T) = (R \cdot S) \cup (R \cdot T)$
    \end{claim}

    \begin{proof}
        Suppose $(a,c) \in R \cdot (S \cup T)$.\pause
        So there exists some $b$ such that $(a,b) \in R$ and $(b,c) \in S$ or $(b,c) \in T$.\pause
        \begin{itemize}
            \item If $(b,c) \in S$ then $(a,c) \in R \cdot S$ holds. \pause
            \item If $(b,c) \in T$ then $(a,c) \in R \cdot T$ holds. \pause
        \end{itemize}
        Thus we have $(a,c) \in (R \cdot S) \cup (R \cdot T)$.\\ \pause
        Conversely, suppose $(a,c) \in (R \cdot S) \cup (R \cdot T)$. \pause
        \begin{itemize}
            \item If $(a,c) \in (R \cdot S)$ then we have $(a,b) \in R$ and $(b,c) \in S$ for some $b$. \pause
            So we obtain $(b,c) \in S \cup T$. Thus we have $(a,c) \in R \cdot (S \cup T)$. \pause
            \item If $(a,c) \in (R \cdot T)$ then similary we have $(a,c) \in R \cdot (S \cup T)$.
        \end{itemize}
    \end{proof}
\end{frame}

\newcommand{\yes}{{\color{green}$\checkmark$}}
\newcommand{\no}{{\color{red}$\times$}}

\begin{frame}
    \frametitle{A.6}

    \begin{table}
        \centering
    \begin{tabular}{r|ccccc}
        & $>$ on $\mathbb{N}$ & $\geq$ on $\mathbb{Z}$ & $\emptyset$ on $\mathbb{N}$ & $\mathrm{id}_{\mathbb{N}}$ & $\emptyset$ on $\emptyset$ \\ \hline
        reflexive     & \no  & \yes & \no  & \yes & \yes \\
        irreflexive   & \yes & \no  & \yes & \no  & \yes \\
        transitive    & \yes & \yes & \yes & \yes & \yes \\
        symmetric     & \no  & \no  & \yes & \yes & \yes \\
        asymmetric    & \yes & \no  & \yes & \no  & \yes \\
        antisymmetric & \yes & \yes & \yes & \yes & \yes \\
    \end{tabular}
    \end{table}
\end{frame}

\newcommand{\RL}[2]{#1\, R\, #2}
\begin{frame}
    \frametitle{$\emptyset$ on $A$}
    \begin{itemize}
    \pause
    \item reflexive     : $a\in A \Rightarrow \RL{a}{a}$ \pause \quad \no \; \pause because $(a,a) \notin \emptyset$
    \pause
    \item irreflexive   : $a \in A \Rightarrow \RL{a}{a} \notin R$ \pause \quad \yes \; \pause because $(a,a)\notin \emptyset$
    \pause
    \item transitive    : $\RL{a}{b}$ and $\RL{b}{c}$ $\Rightarrow$ $\RL{a}{c}$ \pause \yes\; \pause because $(a,b) \notin \emptyset$
    \pause
    \item symmetric     : $\RL{a}{b} \Rightarrow \RL{b}{a}$ \quad \pause \yes\; \pause because $(a,b) \notin \emptyset$
    \pause
    \item asymmetric    : $\RL{a}{b} \Rightarrow \RL{b}{a} \notin R$ \pause \quad \yes\; \pause because $(a,b) \notin \emptyset$
    \pause
    \item antisymmetric : $\RL{a}{b}$ and $\RL{b}{a} \Rightarrow a=b$ \pause \quad \yes\; \pause because $(a,b) \notin \emptyset$
    \end{itemize}
\end{frame}

\begin{frame}
    \begin{claim}[A.7(3)]
        $R$ is transitive $\Leftrightarrow$ $R \cdot R \subseteq R$
    \end{claim}

    \begin{proof}
        \begin{itemize}
            \pause
            \item \underline{\Rightarrow-direction:} \pause
            Suppose $R$ is transitive. \pause
            Let $(a,b) \in R$ and $(b,c) \in R$. \pause
            Then we have $(a,c) \in R \cdot R$. \pause
            Since $R$ is transitive, $(a,c) \in R$ \pause
            \item \underline{\Leftarrow-direction:} \pause
            Suppose $R \cdot R \subseteq R$. \pause
            Assume $(a,b) \in R$ and $(b,c) \in R$. \pause
            Then we have $(a,c) \in R \cdot R$. \pause
            So we obtain $(a,c) \in R$. \pause
            Hence $R$ is transitive.
      \end{itemize}
    \end{proof}
\end{frame}

\begin{frame}
    \begin{claim}[A.7(6)]
        $R$ is antisymmetric $\Leftrightarrow$ $R \cap R^{-1} \subseteq \mathrm{id}_A$
    \end{claim}
    \begin{proof}\pause
        \begin{itemize}
            \item \underline{\Rightarrow-direction:}
                Suppose $R$ is antisymmetric.\pause
                Let $(a,b) \in R \cap R^{-1}$.\pause
                Then we have $(a,b) \in R$ and $(a,b) \in R^{-1}$.\pause
                Since $R$ is antisymmetric, $a=b$. Hence $(a,b) \in \mathrm{id}_A$.\pause
            \item \underline{\Leftarrow-direction:}
                Suppose $R \cap R^{-1} \subseteq \mathrm{id}_A$.\pause
                Let $(a,b) \in R$ and $(b,a) \in R$.\pause
                Then $(a,b) \in R \cap R^{-1}$.\pause
                Thus $(a,b) \in \mathrm{id}_A$.\pause
                Hence $a = b$.
        \end{itemize}
    \end{proof}
\end{frame}

\end{document}
